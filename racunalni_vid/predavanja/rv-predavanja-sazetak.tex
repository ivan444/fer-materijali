\documentclass[11pt,english]{article}
\usepackage[utf8]{inputenc}
\usepackage[croatian]{babel}
\usepackage[T1]{fontenc}
\usepackage{graphicx}
\usepackage{booktabs}
\usepackage{amsmath}
\usepackage{amssymb}
\usepackage{times}
\usepackage[a4paper, left=0.5cm, right=0.5cm, bottom=2cm, top=0.5cm]{geometry}
\usepackage{enumerate}


\begin{document}
\title{Računalni vid}
\date{}
\maketitle

\section{Oblikovanje slike}

\subsection{Primarna slika}
\begin{itemize}
  \item informacija o $2D$ slici, prvenstveno o promjenama svjetlosnih intenziteta, njihovoj distribuciji i organizaciji,
  \item građevni elementi:
  \begin{itemize}
    \item prolasci kroz nulu (\emph{zero-crossings}),
    \item mrlje (\emph{blobs}),
    \item rubni segmenti,
    \item virtualne linije,
    \item grupe/nakupine,
    \item granične linije.
  \end{itemize}
\end{itemize}

\subsection{$2^{1/2}D$ skica}
\begin{itemize}
  \item informacija o orijentaciji i ``gruboj'' dubini vidljivih površina, te konture i diskontinuiteti tih komponenti (u koordinatnom sustavu promatrača),
  \item građevni elementi:
  \begin{itemize}
    \item lokalna orijentacija vidljivih površina,
    \item gruba udaljenost od promatrača,
    \item diskontinuiteti u dubini,
    \item diskontinuiteti u orijentaciji površina.
  \end{itemize}
\end{itemize}

\subsection{Uzorkovanje}
\begin{itemize}
  \item predstavljanje kontinuirane funkcije u digitalnom računalu,
  \item \emph{Shannonov teorem:} ako je Fourierova transformacija funkcije $f(t)$ takva da vrijedi $F(\omega) = 0$ za $|\omega|\geq 2\pi f_c$, funkcija $f(t)$ može biti rekonstruirana ako se njene vrijednosti uzorkuju s $2f_c$ ili većom frekvencijom.
\end{itemize}


\section{Geometrijska svojstva}

Veličina objekta:
$$A = \sum\limits^n_{i=1}\sum\limits^m_{j=1} B[i,j]$$
$A$ -- moment nultog reda.
Moment reda ($p+q$):
$$m_{pq} = \int\limits^{+\infty}_{-\infty}\int\limits^{+\infty}_{-\infty}x^py^q f(x,y) \operatorname{d}x \operatorname{d}y$$
Diskretni slučaj:
$$m_{pq} = \sum\limits^n_{i=1}\sum\limits^m_{j=1} i^p j^q B[i,j]$$
Položaj objekta -- koordinate težišta:
\begin{align*}
\bar x &= \frac{m_{10}}{m_{00}},\\
\bar y &= \frac{m_{01}}{m_{00}}.
\end{align*}
Centralni (središnji) moment stupnja ($p+q$):
$$\mu_{pq} =\sum_{j \in R}\sum_{i \in R} (j-\bar x)^p(i-\bar y)^q B[i,j].$$
Centriranjem koordinatni sustav se pomiče u težište objekta te su u tom sustavu momenti nezavisni od pozicije objekta na slici.\\
Orijentacija objekta ($\theta$):
\begin{itemize}
  \item Orijentacija osi u smjeru izduženja objekta.
  \item Formula za izračun orijentacije prilagođena je za \emph{desno orijentirani} koordinatni sustav (pretvorba lijevo orijentiranog u desno orijentirani: $y$ koordinata se množi s $-1$).
\end{itemize}
$$\theta = \frac{1}{2}\operatorname{atan2}\left (2\mu_{11}, \mu_{20}-\mu_{02} \right ),$$
pri čemu je za $y \ne 0$:
\begin{align*}
  \operatorname{atan2}(y,x) &= \left \{ \begin{array}{ll}
    \theta \cdot \operatorname{sgn}(y),&\quad x>0,\\
    \frac{\pi}{2}\cdot\operatorname{sgn}(y),&\quad x=0,\\
    (\pi-\theta)\cdot\operatorname{sgn}(y),&\quad x<0,
  \end{array}\right.\\
\end{align*}
a za $y = 0$:
\begin{align*}
  \operatorname{atan2(y,x)} &= \left \{ \begin{array}{ll}
    0,&\quad x>0,\\
    \text{nedefinirano},&\quad x=0,\\
    \pi,&\quad x<0,
  \end{array}\right.
\end{align*}
gdje je $\theta$ kut dobiven iz:
$$\operatorname{tg}(\theta) = \left \vert \frac{y}{x}\right \vert.$$
%\begin{align*}
%\mathrm{tg}\left ( 2\theta \right )&= \dfrac{b}{a-c},\\
%a &= \sum\limits^n_{i=1}\sum\limits^m_{j=1} (j')^2 B[i,j],\\
%b &=2\sum\limits^n_{i=1}\sum\limits^m_{j=1} i'\cdot j' B[i,j],\\
%c &= \sum\limits^n_{i=1}\sum\limits^m_{j=1} (i')^2 B[i,j],\\
%j'&= j - \bar x,\quad i' = i - \bar y
%\end{align*}

\section{Binarni algoritmi}
\subsection{Susjedstvo}
\begin{itemize}
  \item 4-susjedstvo $[i,j]$: $[i\pm 1, j]$, $[i,j\pm 1]$
  \item 8-susjedstvo $[i,j]$: $[i\pm 1, j]$, $[i,j\pm 1]$, $[i\pm 1, j\pm 1]$
\end{itemize}

\subsection{Projekcije}
Horizontalna:
  $$H[i] = \sum\limits^m_{j=1}B[i,j]$$
Vertikalna:
  $$V[j] = \sum\limits^n_{i=1}B[i,j]$$
Vrijedi:
  \begin{align*}
    A &= \sum\limits^m_{j=1}V[j] = \sum\limits^n_{i=1}H[i],\\
    \bar y &= \dfrac{\sum\limits^n_{i=1} i H[i]}{A},\\
    \bar x &= \dfrac{\sum\limits^m_{j=1} j V[j]}{A}
  \end{align*}

\subsection{Povezanost}
Slikovni element $p\in S$ je povezan s $q \in S$ ako postoji put od $p$ do $q$ koji se sastoji od slikovnih elemenata iz $S$.

\emph{Povezana komponenta} je skup slikovnih elemenata u kojem je svaki slikovni element povezan s ostalim slikovnim elementima.

\subsection{Filtar veličine}
Uklanjanje svih područja u binarnoj slici koja imaju manje od $T_0$ slikovnih elemenata.

\subsection{Eulerov broj}
$$E=C-H$$
$C$ -- broj komponenti,
$H$ -- broj rupa.

\subsection{Označavanje komponenti}
Nalaženje povezanih komponenti u slici.

\subsubsection{Rekurzivni algoritam}
\begin{enumerate}[\indent K1:\quad]
  \item Nađi prvi element s vrijednošću 1. Dodijeli mu novu oznaku $L$.
  \item Rekurzivno dodijeli oznaku $L$ svim njegovim susjedima.
  \item Kad više nema neoznačenih susjeda s vrijednošću 1, idi na korak $K1$.
\end{enumerate}

\subsubsection{Sekvencijalni algoritam}
\begin{enumerate}[\indent K1:\quad]
  \item Skeniraj sliku s lijeva udesno (s vrha prema dnu).
  \item Ako je slikovni element 1, tada:
  \begin{enumerate}[\indent \indent (a)]
    \item Ako samo jedan od gornjeg i lijevog susjeda imaju oznaku onda kopiraj tu oznaku.
    \item Ako oba imaju istu oznaku onda kopiraj tu oznaku.
    \item Ako oba imaju različite oznake, onda kopiraj oznaku gornjeg i unesi oznake u tablicu ekvivalencija.
    \item U drugim slučajevima dodijeli novu oznaku tom slikovnom elementu i unesi oznaku u tablicu ekvivalencija.
  \end{enumerate}
  \item Ako nema više točaka za razmatranje prijeđi na korak $K2$.
  \item Nađi najmanju vrijednost u ekvivalentnom skupu oznaka iz tablice ekvivalencija.
  \item Skeniraj sliku. Zamijeni svaku oznaku ekvivalencijom najmanje vrijednosti.
\end{enumerate}

\subsection{Granica područja}
Granica povezane komponente $S$ je skup točaka iz $S$ koje su susjedne $\bar S$.

\subsubsection{Algoritam BF}
\begin{enumerate}[\indent K1:\quad]
  \item Provjeri je li objekt ($S$) samo jedan izolirani slikovni element. Ako je, ta točka je ujedno i granica područja. Zaustavi postupak.
  \item Nađi inicijalni par susjednih točaka $c \in S$ i $d \in \bar S$.
  \item Promjeni vrijednost točke $c$ u 3 i točke $d$ u 2.
  \item Neka 8-susjedstvo točke $c$ u smjeru kazaljke na satu, započevši od točke $d$, a završivši s prvim pojavljivanjem 1, 3 ili 4, bude $e_1, e_2, \ldots, e_k$.
  \begin{enumerate}[\indent \indent (a)]
    \item Ako $c=3$, $e_k = 4$ i $e_h=2$ za neki $h<k$, promijeni 3 u 4, 2 u 0 i zaustavi se.
    \item Inače, ako je vrijednost $c$ 1, promijeni vrijednost $c$ u 4. Uzmi $e_k$ kao novu vrijednost $c$ i $e_{k-1}$ kao novu vrijednost $d$, Vrati se na korak 4.
  \end{enumerate}
  Kada se BF zaustavi, slikovni elementi koji imaju vrijednost 4 predstavljaju granicu.
\end{enumerate}

\section{Opis konture}
Kontura je periodična kompleksna funkcija $d(t)$ s vrijednostima iz skupa $\{(x_t + jy_t)\, |\, t = 0, 1, \ldots, T-1\}$.
Eulerova formula:
\begin{align*}
e^{\varphi j} &= \cos \varphi + j \sin \varphi,\\
e^{-\varphi j} &= \cos \varphi - j \sin \varphi
\end{align*}
$d^c(t)$ -- funkcija konture s početnom točkom obilaženja $c$.
Utjecaj početne točke:
$$d(t) = d^c(t+\tau)$$
Utjecaj pomaka:
$$d(t)=d^c(t)+\Delta$$
Utjecaj rotacije:
$$d(t)=e^{j\varphi}d^c(t)$$
Utjecaj uvećanja:
$$d(t)=\alpha d^c(t)$$

\subsection{Opis konture Fourierovim deskriptorima}
Koeficijenti diskretne Fourierove transformacije:
$$F(n)=\dfrac{1}{T} \sum^{T-1}_{t=0} d(t)e^{-j2\pi (nt/T)}, \quad n = 0, 1, \ldots, T-1$$
Inverz:
$$d(t) = \sum^{T-1}_{t=0} F(n)e^{j2\pi (nt/T)}$$

\subsection{Gronlundovi koeficijenti}
Funkcije Fourierovih koeficijenata koje eliminiraju utjecaj izbora početne točke, pomaka, rotacije i uvećanja konture:
$$d_{pq} = \dfrac{F^q(p+1)\cdot F^p(T+1-q)}{F^{(p+q)}(1)}$$

\subsection{Opis konture funkcijama geometrijskih momenata}
Normiranjem dobivamo normirane središnje momente koji su neovisni o položaju i uvećanju objekta:
$$\eta_{pq} = \dfrac{\mu_{pq}}{\mu^r_{00}},$$
pri čemu je $r = ((p+q)/2)+1$ i $(p+q) = 2, 3, \ldots$

Iz normiranih središnjih momenata dobivamo Huove koeficijente:
\begin{align*}
\phi_1 = &\eta_{20} + \eta_{02},\\
\phi_2 = &(\eta_{20} - \eta_{02})^2 + 4\eta_{11}^2,\\
\phi_3 = &(\eta_{30} - 3\eta_{12})^2 + (3\eta_{21} - \eta_{03})^2,\\
\phi_4 = &(\eta_{30} + \eta_{12})^2 + (\eta_{21} + \eta_{03})^2,\\
\phi_5 = &(\eta_{30} -3\eta_{12})(\eta_{30} + \eta_{12}) \cdot
[(\eta_{30} + \eta_{12})^2-3(\eta_{21} + \eta_{03})^2]+\\
&(3\eta_{21}-\eta_{03})(\eta_{21} + \eta_{03}) \cdot
[3(\eta_{30} + \eta_{12})^2 - (\eta_{21} + \eta_{03})^2],\\
\phi_6 = &(\eta_{20} - \eta_{02}) \cdot
[(\eta_{30} + \eta_{12})^2- (\eta_{21} + \eta_{03})^2]+\\
&4\eta_{11}(\eta_{30} + \eta_{12})(\eta_{21} + \eta_{03}),\\
\phi_7 = &(3\eta_{21} - \eta_{03})(\eta_{30} + \eta_{12}) \cdot
[(\eta_{30} + \eta_{12})^2-3(\eta_{21} + \eta_{03})^2]-\\
&(\eta_{30}-3\eta_{12})(\eta_{21} + \eta_{03}) \cdot
[3(\eta_{30} + \eta_{12})^2 - (\eta_{31} + \eta_{03})^2].
\end{align*}

\subsection{Opis konture morfološkim značajkama}
Neovisne su od položaja, pomaka, rotacije i povećanja.

Kompaktnost:
$$k = \dfrac{4\pi A}{O^2},$$
pri čemu je $A$ površina, a $O$ opseg područja. Kompaktnost se mijenja između 0 i 1.

Rastegnutost:
$$e = \dfrac{a}{b},$$
pri čemu je $a$ dužina pravokutnika koji opisuje područje, a $b$ širina pravokutnika.

\section{Segmentacija}
Dijeljenje slike $F[i,j]$ na područja $P_1,P_2, \ldots, P_k$ uz uvjete:
\begin{enumerate}
  \item $\cup^k_{i=1}P_i = F[i,j]$
  \item $P_i \cap P_j = \emptyset$
  \item Sve točke $P_i$ imaju neko zajedničko svojstvo
\end{enumerate}
Vrste segmentacije:
\begin{itemize}
  \item Amplitudna segmentacija:
  \begin{itemize}
    \item uspoređivanje s pragom,
    \item uspoređivanje s pragom u višedimenzionalnom prostoru.
  \end{itemize}
  \item Metode segmentacije područja:
  \begin{itemize}
    \item raspršivanje područja,
    \item stapanje područja,
    \item kombinacija raspršivanja i stapanja,
    \item narastanje područja,
    \item heuristički algoritmi.
  \end{itemize}
  \item Rubna segmentacija:
  \begin{itemize}
    \item rubni operatori koji aproksimiraju gradijentni operator,
    \item rubni operatori s višestrukim maskama,
    \item relaksacijske metode nalaženja rubova.
  \end{itemize}
  \item Nalaženje rubova u teksturi (građi)
\end{itemize}

\subsection{Metoda p-pokrivenosti (p-tile method)}
%  \item Upotrebljava se znanje o površini objekta.
%  \item Na temelju te informacije podijelimo histogram ulazne slike s jednim ili više pragova tako da $p$ postotaka slikovnih elemenata pripada objektu.
\begin{enumerate}[\indent K1:\quad]
  \item Pretpostavimo da objekti zauzimaju $p$ postotaka cijele slike.
  \item Kreni od intenziteta 0 na histogramu i broji slikovne elemente.
  \item Kad postotak prebrojanih bude jednak ili veći od $p$, uzmi zadnju vrijednost intenziteta na histogramu za vrijednost praga.
\end{enumerate}

\subsection{Vršna metoda (mode method)}
Uvjet je da objekti imaju jednake vrijednosti sivih razina dok pozadina ima različite.
\begin{enumerate}[\indent K1:\quad]
  \item Nađi dva najveća lokalna maksimuma u histogramu koji se nalaze na minimalnoj udaljenosti $d$. Pretpostavimo da su oni na sivim vrijednostima $g_i$ i $g_j$.
  \item Nađi najmanju točku u histogramu $H$ između $g_i$ i $g_j$, označi ju s $g_k$.
  \item Odredi šiljatost definiranu sa:
  $$\min\left ( H(g_i), H(g_j) \right ) / H(g_k).$$
  \item Povećaj $d$, idi na $K1$.
  \item Upotrijebi kombinaciju $\left ( g_i, g_j, g_k \right )$ s najvećom ``šiljatosti.'' Vrijednost $g_k$ je dobar prag za izdvajanje objekata koji odgovaraju $g_i$ i $g_j$.
\end{enumerate}
Pristup se može poopćiti za slike koje sadrže veći broj objekata sa različitim srednjim vrijednostima sivih razina.

\subsection{Iterativni postupak određivanja praga}
\begin{enumerate}[\indent K1:\quad]
  \item Izaberi inicijalnu vrijednost praga $T$ (dobar izbor: srednja vrijednost slikovnih elemenata).
  \item Podijeli sliku u dvije grupe, $R_1$ i $R_2$ uporabom praga $T$.
  \item Izračunaj srednju vrijednost $\mu_1$ i $\mu_2$ particija $R_1$ i $R_2$,
  $$\mu_k = \frac{1}{N_k}\sum_{t(i,j)\in R_k} B[i,j],$$
  pri čemu je $N_k$ broj slikovnih elemenata područja $R_k$.
  \item Izaberi novu vrijednost praga:
  $$T = \dfrac{1}{2} \left ( \mu_1 + \mu_2 \right ).$$
  \item Ponavljaj korake $K2$ -- $K4$ sve dok se srednje vrijednosti $\mu_1$ i $\mu_2$ u sukcesivnim iteracijama ne prestanu mijenjati.
\end{enumerate}

\subsection{Adaptivni postupak uspoređivanja s pragom}
Podijeliti sliku na $m \times m$ podslika i izabrati prag $T_{ij}$ za svaku takvu podsliku na temelju histograma $ij$-te podslike ($i \leq i,j \leq m$). Konačna segmentacija slike je unija regija podslika.

\subsection{Metoda dvostrukog praga}
Imamo poznate vrijednosti sivih razina koje pripadaju objektima te dodatne vrijednosti svih razina koje pripadaju i objektu i pozadini.
\begin{enumerate}[\indent K1:\quad]
  \item Izaberi dva praga $T_1$ i $T_2$.
  \item Podijeli sliku na 3 područja:
  \begin{enumerate}[\indent \indent $R_1\;$ -- ]
    \item sadrži sve slikovne elemente s vrijednosti sivih razina manjih od $T_1$,
    \item sadrži sve slikovne elemente s vrijednosti sivih razina između $T_1$ i $T_2$,
    \item slikovni elementi s vrijednošću većom od $T_2$.
  \end{enumerate}
  \item Posjeti svaki slikovni element koji je dodijeljen području $R_2$. Ako  slikovni element ima susjedstvo u $R_1$, raspodijeli ga području $R_1$.
  \item Ponavljaj korak $K3$ dok nema slikovnih elemenata za raspodjelu.
  \item Razvrstaj preostale slikovne elemente područja $R_2$ u područje $R_3$.
\end{enumerate}

\section{Prikaz područja}
Tri osnovna načina prikaza:
\begin{enumerate}
  \item \textbf{Plošni (array representation):}
Za prikaz se koristi ploha iste veličine kao područje u originalnoj slici. Elementi ukazuju na područje kojem slikovni elementi pripadaju.
  \item \textbf{Hijerarhijski:}
  \begin{itemize}
    \item Piramidalni prikaz:
Izvorna slika se sastoji od $n \times n$ slikovnih elemenata (obično vrijedi $n = 2^k$, $k \in \mathbb Z$). Slikovni element na razini $l$ se dobiva kombinacijom nekoliko slikovnih elemenata na razini $l+1$. Cijela slike predočena je slikovnim elementom na razini 0.
    \item Quad-stabla:
Slika se rekurzivno dijeli na četiri područja jednakih veličina.
  \end{itemize}
  \item \textbf{Simbolički:}
    \begin{itemize}
      \item pravokutnik koji omeđuje područje,
      \item središte (centroid) područja,
      \item momenti,
      \item Eulerov broj.
    \end{itemize}
\end{enumerate}

\section{Strukture podataka u postupku segmentacije}
\begin{itemize}
  \item grafovi susjednosti područja (RAG -- Region Adjecency Graph),
  \item slikovna stabla (Picture Tree),
  \item super mreža (Super Grid).
\end{itemize}

\subsection{Grafovi susjednosti područja (RAG)}
\begin{enumerate}[\indent K1:\quad]
%  \item Skeniraj pripadnost elementa u plošnom prikazu i izvedi sljedeće korake za svaki indeks slikovnog elementa $[i,j]$:
%  \item Označi $r_1 = a[i,j]$
%  \item Posjeti susjede $[k,l]$ slikovnog elementa $[i,j]$. Za svakog susjeda izvedi sljedeći korak:
%  \item Neka je $r_2 = a[k,l]$. Ako $r_1 \neq r_2$ pridodaj luk između čvorova $r_1$ i $r_2$ u RAG.
  \item Dodaj vrh s oznakom 0 u RAG (vrh 0 označava vanjsku regiju).
  \item Za svaku regiju dodaj vrh u RAG.
  \item Ako dvije regije imaju 4-susjedne slikovne elemente, poveži njihove vrhove bridom (uzmi u obzir i vrh 0).
\end{enumerate}
Dualni RAG: čvorovi predstavljaju granice a lukovi područja koja su ograničena granicama.
\begin{enumerate}[\indent K1:\quad]
  \item Za svaku stranicu RAG-a (i vanjsku!) dodaj vrh u dualni RAG.
  \item Ako dvije stranice u RAG-u dijele brid, dodaj brid između odgovarajućih vrhova u dualnom RAG-u. Ako stranice dijele $n$ bridova, dodaj $n$-terostruko poveži vrhove u dualnom RAG-u.
\end{enumerate}

\subsection{Slikovna stabla}
Naglašavanju obuhvaćanje jednog područja drugim područjem (gnijedženje područja). Dobivaju se rekurzivnim raspršivanjem slike na područja (rekurzija se zaustavlja kad se dobije područje s konstantnim značajkama).

\subsection{Supermreža}
Koristi se radi točnijeg prikaza granica.
\begin{enumerate}[\indent K1:\quad]
  \item Izvorna slika je dimenzija $N \times N$,
  \item Supermreža je rezolucije $(2N+1) \times (2N+1)$,
  \item Svaki slikovni element je okružen s osam ne-slikovnih elemenata u takvoj super mreži,
  \item Ne-slikovni elementi se koriste za prikaz da li postoji granica ili ne između slikovnih elemenata, te u kojem smjeru se proteže granica.
\end{enumerate}

\section{Raspršivanje i stapanje (Split and Merge)}
\begin{enumerate}[\indent K1:\quad]
  \item Započni s cijelom slikom kao jednim područjem.
  \item Izaberi područje $R$. Ako je $H(R)$ lažno, rasprši područje na četiri potpodručja. $H$ predstavlja sličnost između slikovnih elemenata u području:
  $$H(R_i) = \left \{ \begin{array}{ll}
  \top & \text{ako je varijanca vrijednosti sivih razina mala},\\
  \bot & \text{inače}.
  \end{array}\right .$$
  \item Promatraj bilo koja dva ili više susjednih potpodručja $R_1, R_2, \ldots, R_n$ na slici. Ako je $H(R_1 \cup R_2 \cup \ldots \cup R_n)$ istinito stopi tih n područja u jedno.
  \item Ponavljaj gornje korake sve dok nema daljnjeg raspršivanja ili stapanja.
\end{enumerate}

\section{Narastanje područja}
Grupiranje slikovnih elemenata ili potpodručja u veća područja. Postupak započinje sa skupom inicijalnih jezgara (``sjemenki'' područja) iz kojih područja rastu pridruživanjem susjednih slikovnih elemenata koji imaju slična svojstva.

\section{Stapanje područja}
\begin{enumerate}[\indent K1:\quad]
  \item Oblikuj inicijalna područja u slici uporabom postupka usporedbe s pragom (ili nekim sličnim postupkom).
  \item Priredi RAG za sliku.
  \item Za svako područje u slici izvedi sljedeće korake:
  \begin{enumerate}[\indent \indent a) \;]
    \item Promatraj području susjedno područje i ispitaj da li su područja slična.
    \item Za područja koja su slična stopi ih i modificiraj RAG.
  \end{enumerate}
  \item Ponavljaj korak $K3$ sve dok sva područja (koja se mogu stopiti) ne budu stopljena.
\end{enumerate}
Sličnost susjednih područja:
\begin{enumerate}[\indent \text{Pristup} 1:\quad]
  \item Uspoređivanje srednjih vrijednosti svjetlosnih intenziteta područja --- ako se srednje vrijednosti razlikuju manje od određenog praga, područja su kandidati za stapanje.
  \item Pretpostavlja se da se vrijednosti intenziteta dobivaju na temelju vjerojatnosne distribucije. Ako se atributi dvaju područja dobivaju na temelju iste statističke distribucije -- stopi područja.
\end{enumerate}

\subsection{Postupak stapanja statistički sličnih područja}
Tražimo parametre ($\mu, \sigma$) Gaussove distribucije sivih razina područja $R_1$ i $R_2$. Ako se radi o istoj distribuciji (hipoteza $H_0$) -- područja se stapaju, inače ne (hipoteza $H_1$).
\begin{enumerate}[\indent K1:\quad]
	\item Aproksimacija parametara:
	$$\hat \mu = \frac{1}{n}\sum_{i=1}^n g_i,$$
	$$\hat \sigma^2 = \frac{1}{n}\sum_{i=1}^n \left (g_i-\hat \mu\right ),$$
	pri čemu je $n$ broj slikovnih elemenata područja te $g_i$ vrijednost sive razine.
	\item Omjer vjerodostojnosti $L$:
	$$L = \frac{p\left (g_1, g_2, \ldots, g_{m_1+m_2} \vert H_1 \right )}{p\left (g_1, g_2, \ldots, g_{m_1+m_2} \vert H_0 \right )} = \frac{\sigma_0^{m_1+m_2}}{\sigma_1^{m_1} \cdot \sigma_2^{m_2}},$$
	pri čemu je:
	$$p(g_i) = \frac{1}{\sqrt{2\pi}\sigma}e^{-\frac{(g_i-\mu)^2}{2\sigma^2}}$$
	te je $\sigma_0$ procijenjeni parametar ako oba područja pripadaju istom objektu, dok su $\sigma_1$ i $\sigma_2$ procijenjeni parametri svakog područja zasebno.
	\item Ako je omjer vjerodostojnosti ispod vrijednosti praga, tada je to jaki argument da postoji samo jedno područje i da se područja $R_1$ i $R_2$ mogu stopiti.
\end{enumerate}

\subsection{Uklanjanje ``slabih'' rubova}
\emph{Slaba granica:} granica za koju vrijedi da je razlika intenziteta s obje strane granice manja od praga $T$. Dva pristupa:
\begin{enumerate}[\indent \text{Pristup} 1:\quad]
	\item Stopi susjedna područja $R_1$ i $R_2$ ako
	$$\frac{W}{S} > T,$$
	gdje je $W$ duljina dijela granice koji je slab, $T$ je prag, a $S$ minimalni opseg dvaju područja ($S = \min(S_1, S_2)$, pri čemu su $S_1$ i $S_2$ opsezi područja $R_1$ i $R_2$).
	\item Stopi susjedna područja $R_1$ i $R_2$ ako
	$$\frac{W}{S} > T,$$
	gdje je $S$ zajednička granica  (vrijednost $T = 0.75$ obično daje dobre rezultate).
\end{enumerate}

\subsection{Raspršivanje područja}
Ako neko svojstvo područja nije konstantno, područje treba biti raspršeno.
\begin{enumerate}[\indent K1:\quad]
  \item Oblikuj inicijalna područja.
  \item Za svako područje na slici rekurzivno izvodi sljedeće korake:
  \begin{enumerate}[\indent \indent a) \;]
    \item Izračunaj varijancu vrijednosti sivih razina u području.
    \item Ako je varijanca veća od praga, rasprši područje uzduž odgovarajuće granice.
  \end{enumerate}
\end{enumerate}
\emph{Granica:} područje se potencijalno dijeli na ``čvrsti'' broj područja jednakih veličina. Odgovarajuća granica određena je jakošću granice (unutar nekog nehomogenog područja).

\subsection{Otsuova metoda izbora praga za segmentaciju sive slike}
Imamo sliku sa $N$ slikovnih elemenata te $L$ sivih razina.
\begin{enumerate}[\indent K1:\quad]
	\item Odredi vjerojatnost pojave $i$-te sive razine:
	$$p_i = \frac{f_i}{N},$$
	$f_i$ je broj slikovnih elemenata sa sivom razinom $i$.
  % START iz auditornih.
  \item Odredi:
  $$\mu_T=\sum_{i=1}^L i\cdot p_i.$$
  \item Za svaku sivu razinu $k$ odredi:
  \begin{align*}
    \omega_k &= \sum_{i=1}^k p_i,\\
    \mu_k &= \sum_{i=1}^k i\cdot p_i,\\
    \sigma_B^2(k) &= \frac{\left( \mu_T\omega_k-\mu_k\right)^2}{\omega_k\cdot\left(1-\omega_k\right)}.
  \end{align*}
  %KRAJ iz auditornih.
%	\item Segmentiramo sliku na dva područja:
%	$$\begin{array}{ll}
%	C_1 & \text{područje sa sivim razinama} [1, 2, \ldots, t],\\
%	C_2 & \text{područje} [t+1, \ldots, L].
%	\end{array}$$
%	Odredimo vjerojatnosne distribucije:
%	\begin{align*}
%	C_1: & \frac{p_1}{\omega_1(t)}, \frac{p_2}{\omega_1(t)}, \ldots, \frac{p_t}{\omega_1(t)},\\
%	& \omega_1(t) = \sum^t_{i=1} p_i,\\
%	C_2: & \frac{p_{t+1}}{\omega_2(t)}, \frac{p_{t+2}}{\omega_2(t)}, \ldots, \frac{p_L}	{\omega_2(t)},\\
%		& \omega_2(t) = \sum^L_{i=t+1} p_i.
%	\end{align*}
%	\item Odredimo srednje vrijednosti područja $C_1$ i $C_2$:
%	\begin{align*}
%	\mu_1 &= \sum^t_{i=1} \frac{i \cdot p_i}{\omega_1(t)},\\
%	\mu_2 &= \sum^L_{i=t+1} \frac{i \cdot p_i}{\omega_2(t)}.
%	\end{align*}
%	\item Odredi srednju vrijednost sivih razina cijele slike, $\mu_T$:
%	$$\mu_T =  \sum^L_{i=1} i \cdot p_i.$$
%	Vrijedi:
%	\begin{align*}
%	\omega_1\mu_1+\omega_2\mu_2 &= \mu_T,\\
%	\omega_1 + \omega_2 &= 1,
%	\end{align*}
%  pri čemu je $\omega_i$ skraćen zapis $\omega_i(t)$.
%	\item Odredi varijancu za područja:
%	\begin{align*}
%	\sigma_1^2 &= \sum^t_{i=1} (i - \mu_1)^2 \cdot \frac{p_i}{\omega_1},\\
%	\sigma_2^2 &= \sum^L_{i=t+1} (i - \mu_2)^2 \cdot \frac{p_i}{\omega_2}.
%	\end{align*}
%	\item Dobrotu praga promatramo kroz omjere varijanci $\sigma_W^2$ (unutar razreda),  $\sigma_B^2$ (između razreda) te  $\sigma_T^2$ (ukupna varijanca).
%	Diskriminantne mjere:
%	$$\lambda = \frac{\sigma_B^2}{\sigma_W^2}, \quad \kappa = \frac{\sigma_T^2}{\sigma_W^2}, \quad \eta  = \frac{\sigma_B^2}{\sigma_T^2}.$$ 
%	Vrijedi:
%	\begin{align*}
%	\sigma_W^2 &= \omega_1 \sigma_1^2 + \omega_2\sigma_2^2,\\
%	\sigma_B^2 &= \omega_1 \omega_2 (\mu_2 - \mu_1)^2,\\
%	\sigma_T^2 &= \sum_{i=1}^L(i-\mu_T)^2p_i.
%	\end{align*}
	\item Odredi optimalni prag:
	$$t^\ast = \operatorname*{arg\,max}_{1\leq k<L} \left \{ \sigma_B^2(k) \right \}.$$
\end{enumerate}
Ako imamo $M$ područja ($C_1, \ldots, C_M$), tada imamo $M-1$ pragova. Vrijedi:
\begin{align*}
	\sigma_B^2 &= \sum_{i=1}^M\omega_k(\mu_k-\mu_T)^2,\\
	\omega_k &= \sum_{i \in C_k} p_i,\\
	\mu_k &= \sum_{i \in C_k} \frac{i\cdot p_i}{\omega_k}.
\end{align*}

\section{Rubna segmentacija}
Segmentacija slike na temelju naglih promjena vrijednosti atributa.
\begin{description}
  \item[Rubna točka:] točka u slici s koordinatama $[i, j]$ na mjestu značajne lokalne promjene svjetlosnog intenziteta.
  \item[Rubni fragment:] $i$ i $j$ koordinate ruba te orijentacija ruba ($\theta$).
  \item[Rubni detektor:] algoritam kojim se generira skup rubnih točaka ili fragmenata.
  \item[Oblici ruba:] stepenasti, u obliku rampe, linijski, u obliku krova ili njihove kombinacije.
\end{description}

\subsection{Gradijent i njegova numerička aproksimacija}
$$\mathbf G[f(x,y)] = \begin{bmatrix}G_x\\ G_y\end{bmatrix} = \begin{bmatrix}\frac{\partial f}{\partial x}\\ \frac{\partial f}{\partial y}\end{bmatrix}$$
Svojstva:
\begin{enumerate}
  \item Vektor $\mathbf G[f(x,y)]$ pokazuje u smjeru maksimalnog iznosa povećanja funkcije $f(x,y)$.
  \item Amplituda gradijenta
  $$G[f(x,y)] = \sqrt{G_x^2 + G_y^2}$$
  jednaka je maksimalnom iznosu povećanja funkcije $f(x,y)$ po jedinici udaljenosti u smjeru $\mathbf G$.
Amplitudu možemo aproksimirati sa $\vert G_x \vert + \vert G_y \vert$ ili $\max(\vert G_x\vert,\vert G_y\vert)$.
\end{enumerate}
Smjer gradijenta:
$$\alpha(x,y) = \operatorname{arctg}\left ( \frac{G_y}{G_x}\right ).$$
Za digitalne slike vrijedi:
\begin{align*}
G_x \cong f[i,j+1]-f[i,j],\\
G_y \cong f[i,j]-f[i+1,j].
\end{align*}
Računanje komponenti od $\mathbf G$ može se izvesti pomoću konvolucijskih maski:
$$G_x = \begin{array}{|c|c|}\hline -1\; & \;~1\;\\ \hline \end{array} \quad \text{i} \quad G_y = \begin{array}{|c|}\hline \;~1\; \\\hline -1\;\\ \hline \end{array}$$
Da bi se parcijalne derivacije po $x$ i po $y$ računale u istoj točki 
koriste se maske dimenzija $2 \times 2$ (interpolirana točka je $\left [i+\frac{1}{2},j+\frac{1}{2}\right ]$:)
$$G_x =
\begin{array}{|c|c|}
\hline
-1\; & \;~1~\;\\ \hline
-1\; & \;~1~\;\\ \hline
\end{array}
\quad \text{i} \quad
G_y =
\begin{array}{|c|c|}
\hline
\;~1\; & \;~1\;\\ \hline
-1\; & -1\;\\ \hline
\end{array}$$
Koraci u postupku detekcije ruba:
\begin{enumerate}[\indent K1:\quad]
  \item Filtriranje.
  \item Pojačanje (gradijentni operator).
  \item Detekcija (uspoređivanje s pragom).
  \item Lokalizacija (procjena položaja ruba na subpixel rezoluciji).
\end{enumerate}

\subsection{Robertsov križni operator}
Aproksimira gradijent u interpoliranoj točki $\left [i+\frac{1}{2},j+\frac{1}{2}\right ]$:
$$G\left [f[i,j]\right ] = \left \vert f[i,j] - f[i+1,j+1]\right \vert + \left \vert f[i+1,j] - f[i,j+1] \right \vert,$$
preko konvolucijskih maski:
$$G\left [f[i,j]\right ] = \left \vert G_x \right \vert + \left \vert G_y \right \vert,$$
pri čemu je:
$$G_x =
\begin{array}{|c|c|}
\hline
\;~1~\; & \;0\;\\ \hline
\;~0~\; & -1\;\\ \hline
\end{array}
\quad \text{i} \quad
G_y =
\begin{array}{|c|c|}
\hline
\;~0~\; & -1\;\\ \hline
\;~1~\; & \;0\;\\ \hline
\end{array}$$

\subsection{Konvolucija}
Kontinuirana funkcija:
$$h(x,y) = f(x,y) \ast g(x,y) =  \int_{-\infty}^{+\infty}\int_{-\infty}^{+\infty} f(x', y')g(x-x', y-y')\operatorname{d}x'\operatorname dy',$$
diskretna funkcija:
$$h[i,j] = f[i,j] \ast g[i,j] = \sum_{k=1}^{n}\sum_{l=1}^m f[k,l]\cdot g[i-k,j-l].$$
%TODO: Skraćeni postupak računanja konvolucije.

\subsection{Sobelov operator}
Koristi masku dimenzija $3 \times 3$ da bi se izbjeglo računanje gradijenta u interpolacijskoj točki:
$$\begin{array}{|c|c|c|}
\hline
\;a_0\; & a_1 & \;a_2\;\\ \hline
a_7 &[i,j]& a_3\\ \hline
a_6 & a_5 & a_4\\ \hline
\end{array}$$
Amplituda gradijenta:
$$M = \sqrt{S_x^2 + S_y^2},$$
pri čemu su:
\begin{align*}
S_x &= (a_2+2a_3+a_4)-(a_0+2a_7+a_6),\\
S_y &= (a_0+2a_1+a_2)-(a_6+2a_5+a_1),
\end{align*}
ili uporabom konvolucijskih maski:
$$S_x =
\begin{array}{|c|c|c|}
\hline
-1\; & \;~0~\; & \;~1~\;\\ \hline
-2\; & \;0\; & \;2\;\\ \hline
-1\; & \;0\; & \;1\;\\ \hline
\end{array}
\quad \text{i} \quad
S_y =
\begin{array}{|c|c|c|}
\hline
~1\; & \;~2~\; & \;~1~\;\\ \hline
0 & 0 & 0\\ \hline
-1\; & -2\; & -1\;\\ \hline
\end{array}$$

\subsection{Prewittov operator}
$$S_x =
\begin{array}{|c|c|c|}
\hline
-1\; & \;~0~\; & \;~1~\;\\ \hline
-1\; & \;0\; & \;1\;\\ \hline
-1\; & \;0\; & \;1\;\\ \hline
\end{array}
\quad \text{i} \quad
S_y =
\begin{array}{|c|c|c|}
\hline
~1\; & \;~1~\; & \;~1~\;\\ \hline
0 & 0 & 0\\ \hline
-1\; & -1\; & -1\;\\ \hline
\end{array}$$

\subsection{Laplacian}
Dvodimenzionalni ekvivalent druge derivacije funkcije $f(x,y)$:
$$\nabla^2 f = \frac{\partial^2f}{\partial x^2} + \frac{\partial^2f}{\partial y^2}.$$
Aproksimacija diferencijama:
\begin{align*}
\frac{\partial^2f}{\partial x^2} &= \frac{\partial G_x}{\partial x} = \frac{\partial ( f[i,j+1]-f[i,j])}{\partial x}\\
&=\frac{\partial f[i,j+1]}{\partial x}-\frac{\partial f[i,j]}{\partial x}\\
&=\left ( f[i,j+2]-f[i,j+1]\right ) - \left ( f[i,j+1]-f[i,j]\right )\\
&=f[i,j+2]-2f[i,j+1]+f[i,j].
\end{align*}
Aproksimacija je centralizirana oko točke $[i,j+1]$. Zamjenom $j$ s $j-1$ dobiva se:
$$\frac{\partial^2 f}{\partial x^2} = f[i,j+1]-2f[i,j]+f[i,j-1].$$
Analogno vrijedi:
$$\frac{\partial^2 f}{\partial y^2} = f[i+1,j]-2f[i,j]+f[i-1,j].$$
Kombiniranjem dobiva se maska:
$$\nabla^2 \approx
\begin{array}{|c|c|c|}
\hline
\;~0~\; & \;~1~\; & \;~0~\;\\ \hline
1 & -4& 1\\ \hline
0 & 1 & 0\\ \hline
\end{array}$$
Središnji element se može naglasiti, primjerice:
$$\begin{array}{|c|c|c|}
\hline
\;~~1~~\; & 4 & \;~~1~~\;\\ \hline
4 &-20\;& 4\\ \hline
1 & 4 & 1\\ \hline
\end{array}$$

\subsection{Laplacian Gaussove maske (LoG)}
\begin{enumerate}[\indent K1:\quad]
  \item Izgladi sliku pomoću Gaussove maske.
  \item ``Pojačaj'' rubove uporabom druge derivacije ($2D$ Laplacian).
  \item Detektiraj mjesta ``prolaska kroz nulu.''
\end{enumerate}
LoG operator:
$$h(x,y) = \left [ \nabla^2 g(x,y) \right ] \ast f(x,y),$$
pri čemu vrijedi:
$$\nabla^2g(x,y) = \left ( \frac{x^2+y^2-2\sigma^2}{4\sigma^2} \right ) \cdot e^{-\frac{x^2+y^2}{2\sigma^2}}.$$

\subsubsection{Izvod Gaussove maske}
Zero-mean Gaussova funkcija:
$$g(x)=e^{-\frac{x^2}{2\sigma^2}}.$$
Dvodimenzionalna zero-mean diskretna Gaussova funkcija:
$$g[i,j] = e^{-\frac{i^2+j^2}{2\sigma^2}},$$
uz normalizaciju:
$$\frac{g[i,j]}{c} = e^{-\frac{i^2+j^2}{2\sigma^2}}.$$
Za izabranu vrijednost $\sigma^2$ dobivamo $n \times n$ masku (jezgro),
$$n \geq 2\sigma^2.$$
\begin{enumerate}[\indent K1:\quad]
  \item Za određene vrijednosti $n$ i $\sigma^2$ izračunaj $n \times n$ vrijednosti $g[i,j]$.
  \item Uzmi najmanju vrijednost $g_{min}[i,j]$ i nađi $c$ tako da vrijedi: $g_{min}[i,j] \cdot c = 1$.
  \item Pomnoži sve vrijednosti matrice sa $c$ (tako da matrica ima samo cjelobrojne vrijednosti).
\end{enumerate}
\textsc{Napomena:} Nakon konvolucije slikovni elementi moraju biti normalizirani!
\begin{align*}
K &= \sum_i\sum_j g[i,j],\\
h[i,j] &= \frac{1}{K} \cdot \left ( I[i,j] \ast g]i,j] \right ).
\end{align*}

\subsection{Cannyev detektor ruba}
\begin{enumerate}[\indent K1:\quad]
  \item Izgladi sliku Gaussovim filtrom.
  \begin{itemize}
    \item Sliku $I[i,j]$ konvoluiramo sa Gaussovim filtrom $G[i,j:\sigma]$:
    $$S[i,j] = G[i,j:\sigma] \ast I[i,j].$$
  \end{itemize}
  \item Izračunaj amplitudu i orijentaciju gradijenta uporabom aproksimacije parcijalne derivacije, tj.\ aproksimiramo gradijent ``izglađene'' slike $S[i,j]$.
  \item Uporabi potiskivanje nemaksimalnih vrijednosti (nonmaxima suppression) na slici amplituda gradijenta.
  \begin{itemize}
    \item $M[i,j]$ je slikovno polje amplituda gradijenta.
    \item Da bi se našli rubovi, široke linije u $M$ se moraju stanjiti tako da ostanu samo točke s najvećim lokalnim promjenama (non-maxima suppression).
    \item Postupkom potiskivanja ne-maksimalnih vrijednosti linije u slikovnom polju amplituda $M[i,j]$ se stanjuju potiskivanjem vrijednosti uzduž linije gradijenta (i to onih koje nisu vršne vrijednosti).
    \item Postupak započinje redukcijom kuta gradijenta $\theta[i,j]$ u jedan od četiri sektora.
    $$\rho[i,j] = \operatorname{sector}\left(\theta [i,j]\right ).$$
    \item Algoritam prolazi $3 \times 3$ susjedstvo uzduž (i poprijeko) slikovnog polja $M[i,j]$. U svakoj točki -- centralnom elementu $M[i,j]$ susjedstva uspoređuje vrijednosti njegovih dvaju susjeda (uzduž linije gradijenta zadane vrijednosti sektora $\rho[i,j]$).
    \item Ako je amplituda $M[i,j]$ u centru takva da nije veća od oba susjeda uzduž linije gradijenta, tada $M[i,j]$ postavlja u 0.
    \item Rezultat:
    $$N[i,j] = \operatorname{nms}\left( M[i,j],\rho[i,j]\right).$$
    Vrijednosti različite od 0 u $N[i,j]$ odgovaraju (potencijalnim) rubovima.
  \end{itemize}
  \item Uporabi algoritam dvostrukog praga za detekciju i povezivanje rubova.
  \begin{itemize}
    \item Izabiru se dva praga, $\tau_1$ i $\tau_2$, pri čemu vrijedi:
    $$\tau_2 \approx 2\tau_1.$$
    \item Usporedbom s pragom generiraju se dvije slike rubova, $T_1$ i $T_2$.
    \item Ide se po slici $T_2$ i kad se dođe do kraja konture traži se u slici $T_1$ (na lokacijama 8-susjedstva) rub koji se može povezati u konturu.
  \end{itemize}
\end{enumerate}

\subsection{Pronalaženje granica}
Metode:
\begin{itemize}
  \item Metoda pristajanja s krivuljom -- upotrebljava se a priori informacija o obliku granice.
  \item Metoda podjeli-pa-vladaj -- upotrebljava se informacija o postojanju ruba između dvije konačne točke.
  \item Hughova metoda -- znanje o mjestu granice je vrlo malo, ali njen oblik može se opisati u parametarskom obliku i ima jednostavan analitički oblik.
  \item Uopćena Hughova metoda -- za granice koje nemaju jednostavan analitički oblik ali imaju svojstvenu konturu koja se ne može opisati jednostavnim analitičkim oblikom.
  \item Heuristički algoritam pretraživanja grafova.
\end{itemize}

\subsubsection{Metoda pristajanja s krivuljom}
Tražimo polinom $g(x)$ tako da je greška između skupova $\{(x_i,y_i)\}$ i $\{(x_i,g(x_i))\}$ minimalna. Greška se definira sa:
$$\varepsilon = \sum_{i=1}^M\left [ y_i - g(x_i) \right ]^2.$$
Tražimo koeficijente $\mathbf a$ u obliku vektora:
$$\mathbf X \mathbf a = \mathbf g(\mathbf x),$$
odnosno:
$$\begin{bmatrix}
1 & x_1 & x_1^2 & \cdots & x_1^N\\
1 & x_2 & x_2^2 & \cdots & x_2^N\\
& \vdots & & \ddots & \vdots\\
1 & x_M & x_M^2 & \cdots & x_M^N
\end{bmatrix}
\cdot
\begin{bmatrix}
a_0\\
a_1\\
\vdots\\
a_M
\end{bmatrix}
=
\begin{bmatrix}
g(x_0)\\
g(x_1)\\
\vdots\\
g(x_M)
\end{bmatrix}$$
Traženi skup težinskih koeficijenata dan je izrazom:
$$\mathbf a = \mathbf X^- \mathbf g(\mathbf x),$$
pri čemu je $\mathbf X^-$ generalizirani inverz, vrijedi:
$$\mathbf X^- = \left ( \mathbf X^\top\mathbf X\right )^{-1}\mathbf X^\top.$$
Za slučaj $M = N$ vrijedi $\mathbf X^- = \mathbf X^{-1}$.

\subsubsection{Metoda podjeli pa vladaj}
\begin{enumerate}[\indent K1:\quad]
  \item Zadana su dva krajnja slikovna elementa kao krajnje točke granice. Povuci liniju između te dvije točke.
  \item Za svaki slikovni element koji je rubni element i nalazi se u području granice izračunajte pravokutne udaljenosti do linije. Ako su sve te udaljenosti manje od unaprijed zadanog praga -- algoritam završava.
  \item Slikovni element -- rubni element čija je udaljenost najveća i premašuje prag postaje \emph{prijelomna točka}. Zamjeni prijašnji linijski segment s novim linijskim segmentom.
  \item Postupak se rekurzivno provodi za nove segmente.
\end{enumerate}

\subsubsection{Parametarski prostor}
Jednadžba pravca u parametarskom prostoru (prostor $\varrho$-$\theta$):
$$x\cos \theta + y \sin \theta - \varrho = 0,$$
pri čemu je $\varrho$ udaljenost pravca od ishodišta, a $\theta$ kut što ga $\varrho$ čini sa osi $x$.

\subsubsection{Hughova metoda}
\begin{enumerate}[\indent K1:\quad]
  \item Kvantiziraj parametarski prostor između minimalnih i maksimalnih vrijednosti $\varrho$ i $\theta$.
  \item Oblikuj polje brojila $C(\varrho,\theta)$ i inicijaliziraj vrijednost brojila na 0. Svako brojilo odgovara jednoj ćeliji kvantiziranog parametarskog prostora.
  \item Za svaki slikovni element $(x,y)$ u slici koja je dobivena uporabom rubnog operatora povećaj za 1 sva brojila u polju brojila uzduž krivulje u parametarskom prostoru. Krivulja u parametarskom prostoru predstavlja porodicu krivulja koja se u prostoru $x$-$y$ može povući u tom slikovnim elementu.
  \item Lokalni maksimum vrijednosti brojila u polju brojila odgovara slikovnim elementima koji u izvornoj slici leže na traženoj krivulji. Vrijednosti brojila daju broj slikovnih elemenata na krivulji u izvornoj slici.
\end{enumerate}

\subsubsection{Poopćeni Hughov algoritam}
\begin{itemize}
  \item Izaberimo neku točku $(x_c,y_c)$ u silueti i proglasimo je referentnom točkom.
  \item Povucimo dužinu između $C$ i nekog graničnog slikovnog elementa $(x,y)$.
  \item U tom slikovnom elementu izračunajmo smjer gradijenta.
  \item Koordinate referentne točke mogu se izraziti kao funkcija smjera gradijenta i koordinata graničnog slikovnog elementa:
  \begin{align*}
    x_c = x+r(\theta_i)\cos[\alpha_i(\theta_i)],\\
    y_c = y+r(\theta_i)\sin[\alpha_i(\theta_i)],
  \end{align*}
  pri čemu je $r$ udaljenost referentne točke i graničnog elementa, $\alpha_i$ kut koji tvori $r$ s osi $x$.
  \item Za poznati oblik siluete, orijentaciju i veličinu objekta, neposredno prije traženja računamo tablicu $\mathbf R$ koja ima vrijednosti:
  \begin{itemize}
    \item Smjer gradijenta u svim slikovnim elementima na granici objekta (poznatog oblika siluete).
    \item Smjeru gradijenta pridruženi parovi $(r,\alpha)$.
  \end{itemize}
  Primjer tablice:
  \begin{center}
    \begin{tabular}{cc}
      \toprule
      Smjer gradijenta & Parovi $\mathbf r = (r,\alpha)$\\
      \midrule
      $\theta_1$ & $r_1^1$, $r_2^1$, $\ldots$, $r_{n_1}^1$\\
      $\theta_2$ & $r_1^2$, $r_2^2$, $\ldots$, $r_{n_2}^2$\\
      $\vdots$ & $\vdots$\\
      $\theta_m$ & $r_1^m$, $r_2^m$, $\ldots$, $r_{n_m}^m$\\
      \bottomrule
    \end{tabular}
  \end{center}
\end{itemize}


\begin{enumerate}[\indent K1:\quad]
  \item Formirajte tablicu $\mathbf R$ za oblik koji se traži.
  \item Oblikujte polje brojila $C(x_{c_{min}}:x_{c_{max}}:y_{c_{min}}:y_{c_{max}})$ koje odgovaraju mogućim lokacijama referentne točke. Inicijalizirajte vrijednosti brojila na 0.
  \item Za svaki rubni element u slici:
  \begin{enumerate}[\indent \indent (a)]
    \item Izračunajte smjer gradijenta $\theta$.
    \item Izračunajte moguće koordinate referentne točke. Izračunati smjer gradijenta u koraku (a) određuje redak tablice $\mathbf R$ iz kojeg se uzimaju vrijednosti parova $r$ i $\alpha$:
    \begin{align*}
      x_c = x+r(\theta)\cos[\alpha(\theta)],\\
      y_c = y+r(\theta)\sin[\alpha(\theta)].
    \end{align*}
    \item Povećajte vrijednost brojila u polju brojila za 1:
    $$C(x_c,y_c) = C(x_c,y_c)+1$$
  \end{enumerate}
  \item Moguće lokacije traženog oblika u slici su određene maksimumima u polju $C$.
\end{enumerate}

\end{document}
