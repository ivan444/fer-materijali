\documentclass[12pt,english]{article}
\usepackage[utf8]{inputenc}
\usepackage[croatian]{babel}
\usepackage[T1]{fontenc}
\usepackage{graphicx}
\usepackage{booktabs}
\usepackage{amsmath}
\usepackage{amssymb}
%\usepackage{times}
\usepackage{lmodern}
\usepackage[a4paper, left=0.5cm, right=1.0cm, bottom=2cm, top=0.5cm]{geometry}
\usepackage{enumerate}

\newcommand{\X}{\mathbf X}
\newcommand{\N}{\mathbf N}
\newcommand{\A}{\mathbf A}
\newcommand{\MP}{\mathbf P}
\newcommand{\E}{\mathbb E}
\newcommand{\En}{\mathbf E}
\newcommand{\vertS}{\; \vert \;}
\newcommand{\di}{\mathrm d}

\begin{document}
\title{Stohastički procesi}
\date{}
\maketitle

\section{Funkcije izvodnice}
Funkcija izvodnica slučajne varijable $\mathbf X$:
$$\psi_\mathbf{X}(z) = \mathbb{E} (z^\mathbf X) =
\sum_{n=0}^\infty p_n z^n, \quad p_n = P(\mathbf X = n)$$

\subsection{Neke razdiobe i funkcije izvodnice}
\begin{description}
  \item[Geometrijska] (vrijeme do prve realizacije događaja)
  $$\mathbf X \sim \mathcal G(p), \quad p_k = P(\mathbf X = k) = pq^{k-1}, \quad q = 1-p.$$
  Izvodnica: $\psi(z) = \frac{p}{1-qz}$.

  \item[Binomna] (broj realizacija u $n$ pokusa)
  $$\mathbf X \sim \mathcal B(n,p), \quad p_k = P(\mathbf X = k) = {n \choose k}p^k q^{n-k}, \quad q = 1-p.$$
  Izvodnica: $\psi(z) = (pz+q)^n$.

  \item[Poissonova] (broj realizacija u nekom vremenu)
  $$\mathbf X \sim \mathcal P(\lambda), \quad p_k = P(\mathbf X = k) = \frac{\lambda^k}{k!} e^{-\lambda}.$$
  Izvodnica: $\psi(z) = e^{\lambda (z-1)}$.

\end{description}

\subsection{Očekivanje i disperzija}
Očekivanje:
$$\psi_\mathbf X'(1) = \mathbb E(\mathbf X)$$
Izvod:
$$\psi'(z) = \sum_{n=0}^\infty n z ^{n-1}p_n,$$
za $n \geq 1$ i $z = 1$ vrijedi:
$$\psi'(1) = \sum_{n=1}^\infty n p_n = \mathbb E(\mathbf X).$$
Disperzija:
$$\mathbb D(X) = \mathbb E(\mathbf X^2) - \mathbb E(\mathbf X)^2 = \psi''(1)+\psi'(1)-\psi'(1)^2.$$

\subsection{Suma slučajnih varijabli}
Ako je $\mathbf S = \mathbf X_1 + \mathbf X_2 + \ldots + \mathbf X_n$, tada vrijedi:
$$\psi_\mathbf S (z) = \psi_{\mathbf X_1}(z) \cdot \psi_{\mathbf X_2}(z) \cdots \psi_{\mathbf X_n}(z).$$
Izvod (kraći):
$$\mathbb E(z^{\mathbf X + \mathbf Y}) = \mathbb E (z^\mathbf X \cdot z^\mathbf Y) = \mathbb E(z^\mathbf X) \cdot \mathbb E(z^\mathbf Y).$$

\subsection{Slučajni zbroj slučajnih varijabli}
Ako je $\mathbf S = \mathbf X_1 + \mathbf X_2 + \cdots + \mathbf X_{\mathbf N}$, pri čemu je $\mathbf N$ slučajna varijabla, a $\mathbf X_i$ slučajne varijable jednake razdiobe, tada vrijedi:
$$\psi_{\mathbf S}(z) = \psi_{\mathbf N}(\psi_{\mathbf X}(z)).$$

\subsection{Polinomijalna razdioba}
Imamo niz od $n$ nezavisnih pokusa. U svakom pokusu se može ostvariti jedan od disjunktnih događaja $A_1, A_2, \ldots, A_r$. Vjerojatnost da se $A_1$ ostvario $n_1$ put, $A_2$ $n_2$ puta, \ldots, $A_r$ $n_r$ puta ($n_1+n_2+\cdots+n_r=n$) dana je sa:
$$p_{n_1,n_2,\ldots,n_r}=\frac{n!}{n_1!n_2!\cdots n_r!} \cdot p_1^{n_1} p_2^{n_2} \cdots p_r^{n_r}.$$
Funkcija izvodnica:
$$\psi(u_1,u_2,\ldots,u_r) = (p_1u_1+\cdots+p_ru_r)^n.$$

\subsection{Neke sume}
\begin{align*}
  &\sum_{n=0}^\infty \frac{a^n}{n!}  = e^a, \\
  &\sum_{k=0}^n {n \choose k } p^k q^{n-k} = (p+q)^n,\\
  &\sum_{n_1+n_2+\cdots+n_r=n} \frac{n!}{n_1!n_2!\cdots n_r!} \cdot p_1^{n_1}p_2^{n_2}\cdots p_r^{n_r} = (p_1+p_2+\cdots+p_r)^n,\\
  &\sum_{n=0}^\infty x^n = \frac{1}{1-x}, \quad |x| < 1,\\
  &\sum_{k=0}^\infty (k+1)x^k = \frac{1}{(1-x)^2},\\
  &\sum_{k=0}^\infty (-1)^k {n+k-1 \choose k } x^k = (1+x)^{-n}, \quad \text{jer vrijedi: }
  {-n \choose k} = (-1)^k{n+k-1 \choose k}.
\end{align*}

\section{Slučajno pomicanje}
Slučajno pomicanje je niz nezavisnih slučajnih varijabli:
$$\mathbf X_0 = x_0, \quad \mathbf X_n = x_0 + \mathbf Y_1 + \mathbf Y_2 +\cdots + \mathbf Y_n, \quad n \geq 1,$$
pri čemu je
$$\mathbf Y_n \sim \left( \begin{array}{lr}
-1 & 1\\
q_n & p_n
\end{array} \right),
\quad p_n+q_n = 1.$$

\subsection{Vjerojatnost gubitka}
Vjerojatnost gubitka ako se kreće s $n$ žetona. Rubni slučajevi: pobjeda -- skupljanje $s$ žetona; gubitak -- gubljenje svih žetona. Vjerojatnost pobjede u jednoj igri je $p$, a gubitka $q$ ($q = 1-p$). Vrijedi:
\begin{align*}
a_n &= pa_{n+1}+qa_{n-1}, \quad n=1, 2, \ldots, s-1\\
a_n &= \lambda^n\\
\lambda^n &= p\lambda^{n+1}+q\lambda^{n-1}\\
\lambda &= p\lambda^2 + q \; \Rightarrow \; \lambda_1 = 1,\; \lambda_2 = \frac{q}{p},\; q \neq p\\
a_n &= C_1 + C_2 \left( \frac{q}{p} \right)^n,\; q \neq p,
\end{align*}
odnosno
\begin{align*}
a_n &= C_1 + C_2n,\; q = p,\; \lambda_1 = \lambda_2 = 1.
\end{align*}
Konstante $C_1$ i $C_2$ se dobivaju iz rubnih slučajeva.

Za poopćeno slučajno pomicanje vrijedi:
$$\mathbf Y_n \sim \left( \begin{array}{lcr}
-1 & 0 & 1\\
q_n & r_n & p_n
\end{array} \right),
\quad p_n+r_n+q_n = 1.$$
\begin{align*}
a_n &= pa_{n+1}+ra_n+qa_{n-1}, \quad n=1, 2, \ldots, s-1\\
a_n(1-r) &= pa_{n+1}+qa_{n-1},\\
a_n &= p'a_{n+1}+q'a_{n-1}, \quad p' = \frac{p}{1-r},\; q' = \frac{q}{1-r}.
\end{align*}

\subsection{Očekivano trajanje igre}
\begin{align*}
d_n &= p(d_{n+1}+1)+q(d_{n-1}+1), \quad n=1, 2, \ldots, s-1\\
d_n &= pd_{n+1}+qd_{n-1} + 1.
\end{align*}
Homogena:
\begin{align*}
d_n &= pd_{n+1}+qd_{n-1},\\
0 &= \lambda^2p - \lambda + q.
\end{align*}
Partikularna:
\begin{align*}
d_n &= \alpha n,\; p \neq q \quad (\text{odnosno: } d_n =Cn^2,\; p=q)\\
0 &= \lambda^2p - \lambda + q,\\
p\alpha(n+1) &= \alpha n - q\alpha(n-1)-1,
\end{align*}
uvrštavanjem rubnih slučajeva, dobiva se:
$$\alpha = \frac{1}{q-p}.$$
Rješenje je:
$$d_n = d_H + d_P,$$
za $p \neq q$ vrijedi:
$$d_n = C_1 + C_2\left ( \frac{q}{p}\right)^n + \frac{n}{q-p}.$$

\section{Markovljevi lanci}
Lanac $\mathbf X_1, \mathbf X_2, \ldots$ je markovljev ako za sve izbore stanja $i_1, i_2, \ldots, i_n$ vrijedi:
$$P\left(\mathbf X_{n+1} = i_{n+1} \;|\; \mathbf X_n = i_n, \ldots, \mathbf X_0 = i_0\right) = P\left(\mathbf X_{n+1} = i_{n+1} \;|\; \mathbf X_n = i_n\right).$$

\subsection{Matrica prijelaznih vrijednosti}
$\Pi$ je matrica prijelaznih vrijednosti, te vrijedi:
\begin{align*}
\Pi &= (p_{ij}),\\
p_{ij} &= P\left(\mathbf X_{n+1} = j \;|\; \mathbf X_n = i\right).
\end{align*}
Svojstvo homogenosti: $p_{ij}  = P(\mathbf X_{n+1} = j \;|\; \mathbf X_n = i)= P(\mathbf X_1 = j \;|\; \mathbf X_0 = i)$.

\subsection{Jednadžba markovljevog lanca}
$$p(n) = p(0)\Pi^n,$$
također vrijedi:
$$p(n) = p(n-1)\Pi.$$
Vektor $p(0)$ je vektor početnih stanja, odnosno razdioba slučajne varijable $\mathbf X_0$.

\subsection{Ergodički teorem}
Ako postoji $n$ za koji su svi elementi $\Pi^n$ pozitivni tada za svaki $j$ postoji:
$$\pi_j = \lim_{n \rightarrow \infty} p_{ij}(n).$$

\subsection{Stacionarne vjerojatnosti}
Određuju vjerojatnost da će u nekom dalekom trenutku lanac biti u nekom određenom stanju. Mogu se interpretirati kao prosječni dio vremena koje lanac provodi u nekom stanju. Vrijedi:
\begin{align*}
\Pi^\top\pi &= \pi \quad \Rightarrow \quad \left( \mathbf I - \Pi^\top \right) = \mathbf 0,\\
\sum_j \pi_j &= 1
\end{align*}

\subsection{Klasifikacija stanja}
\begin{itemize}
  \item Stanje $j$ je dostižno iz $i$ ($i \rightarrow j$) ako se u $n$ koraka može doći iz $i$ u $j$.
  \item Stanje $i$ je bitno ako za svaki $j$ za koji vrijedi $i \rightarrow j$ također vrijedi $j \rightarrow i$ (kraće: $i \leftrightarrow j$).
  \item $H$ je bitni skup, te vrijedi:
  \begin{itemize}
    \item $\forall i \in H$, $i$ je bitan.
    \item $\forall i,j \in H$ vrijedi $i \leftrightarrow j$.
    \item Ako $i \in H$ te $j \not\in H$ vrijedi $i \not\rightarrow j$.
  \end{itemize}
\end{itemize}

\section{Stohastički procesi}
Stohastički proces $\mathbf X$ je familija slučajnih varijabli $\mathbf X_t$:
$$\mathbf X = \{\mathbf X_t, t \in T\},\quad \mathbf X : T \times \Omega \rightarrow S.$$
$\mathbf X_t$ opisuje stanje procesa $\mathbf X$ u trenutku $t$, a $S$ je skup stanja unutar kojeg proces poprima vrijednosti ($S \subset \mathbb C$).

Slučajna varijabla je preslikavanje $\mathbf X : \Omega \rightarrow \mathbb R$. Poprima vrijednost $\mathbf X(\omega)$ za svaku realizaciju elementarnog događaja $\omega$.

Trajektorija je funkcija realne varijable $t \mapsto \mathbf X(t,\omega)$ pri čemu je događaj $\omega \in \Omega$ fiksan.

\subsection{Markovljevi procesi}
$\mathbf X$ je Markovljev proces ako za sve $t_1 < t_2 < \ldots < t_n < t$ vrijedi:
$$P(a<\mathbf X_t < b \;\vert\; \mathbf X_{t_1} = x_1, \mathbf X_{t_2} = x_2, \ldots, \mathbf X_{t_n} = x_n) = P(a < \mathbf X_t < b \;\vert\; \mathbf X_{t_n} = x_n).$$

\subsection{Procesi s nezavisnim prirastima}
Za proces $\mathbf X$ kažemo da je proces s nezavisnim prirastima ako su za sve $t_1 < t_2 < \ldots < t_n$ slučajne varijable $\mathbf X_{t_2} - \mathbf X_{t_1}$, $\mathbf X_{t_3} - \mathbf X_{t_2}$, \ldots, $\mathbf X_{t_n} - \mathbf X_{t_{n-1}}$ nezavisne.

\subsection{Stacionarni procesi}
\begin{itemize}
  \item $\mathbf X$ je stacionaran u \emph{užem} smislu ako za svaki $h$ slučajni vektori $\left( \mathbf X_{t_1}, \mathbf X_{t_2}, \ldots, \mathbf X_{t_n}\right)$ i  $\left( \mathbf X_{t_1+h}, \mathbf X_{t_2+h}, \ldots, \mathbf X_{t_n+h}\right)$ imaju jednaku razdiobu. To su procesi čije su konačnodimenzionalne razdiobe invarijantne na pomake u vremenu.

  \item $\mathbf X$ je stacionaran u \emph{širem} smislu ako vrijedi:
  \begin{enumerate}
    \item očekivanje je konstantno: $m(t) = \text{const}$,
    \item korelacijska funkcija $R(t,s)$ ovisi samo o razlici vremena $t-s$.
  \end{enumerate}
\end{itemize}

\subsection{Svojstva stohastičkih procesa}
\begin{description}
  \item[Moment prvog reda:]
  $$m(t) := \mathbb E\left[ \mathbf X_t\right] = \int_{-\infty}^\infty xf_t(x)\mathrm{d}x.$$
  \item[Korelacijska funkcija:]
  $$R(t,s) := \mathbb E\left[\mathbf X_t \mathbf X_s\right] = \int_{-\infty}^\infty \int_{-\infty}^\infty x_1x_2f_{t,s}(x_1,x_2)\mathrm dx_1\mathrm dx_2.$$
  \item[Kovarijacijska funkcija:]
  $$C(t,s) := \mathbb E\left[(\mathbf X_t - m(t))(\mathbf X_s-m(s))\right] = \mathbb E\left[\mathbf X_t\mathbf X_s\right] -m(t)m(s) = R(t,s)-m(t)m(s).$$
  \item[Disperzija slučajne varijable $\mathbf X_t$:]
  $$\mathbb D\left[\mathbf X_t\right] = \mathbb E\left[\mathbf X_t^2\right] - m(t)^2 = R(t,t) - m(t)^2.$$
\end{description}
Ako je $\mathbf X$ stacionaran u \emph{užem} smislu, onda vrijedi:
  $$m(t+h) := \mathbb E\left[ \mathbf X_{t+h}\right] = \int_{-\infty}^\infty xf_{t+h}(x)\mathrm{d}x = \int_{-\infty}^\infty xf_{t}(x)\mathrm{d}x = \mathbb E\left[\mathbf X_t\right] = m(t).$$
Za stacionarni proces također vrijedi:
$$R(t+h,s+h) := \mathbb E\left[\mathbf X_{t+h} \mathbf X_{s+h}\right] = \mathbb E\left[\mathbf X_t \mathbf X_s\right] = R(t,s).$$

\section{Poissonov proces}
Registrira broj realizacija događaja $A$ i trenutke u kojima se događaj zbio.
$\mathbf N(s,t)$ je slučajna varijabla koja mjeri broj realizacija događaja $A$ u intervalu $[s,t]$.

\begin{description}
  \item[Svojstva:]\hfill
    \begin{enumerate}
      \item \emph{Odsustvo pamćenja.} $\mathbf N(s,t)$ ne ovisi o pojavljivanju događaja $A$ prije trenutka $s$.
      \item \emph{Homogenost u vremenu.} $\mathbf N(s,t)$ ovisi samo o duljini intervala $t-s$.
      \item \emph{Regularnost.} U intervalu infinitezimalne duljine $h$, vjerojatnost pojave samo jednog događaja je $\lambda h + o(h)$, a više od jednog događaja $o(h)$.\footnote{Parametar $\lambda$ opisuje \emph{gustoću realizacija} događaja $A$, a za $o(h)$ vrijedi: $\lim\limits_{h\rightarrow 0} \frac{o(h)}{h} = 0$.}
    \end{enumerate}
  Vrijedi: $\mathbf N(s,t) \sim \mathbf N(s+h,t+h)$, $\forall h$. Za $h=-s$: $\mathbf N(s,t) \sim \mathbf N(0,t-s) \mapsto \mathbf N_{t-s}$.

  Za $t>s$ vrijedi:
  $\N_t - \N_s = \N(0,t) - \N(0,s) \sim \N(0,s)+\N(s,t)-\N(0,s) = \N(s,t).$

  \item[Jednodimenzionalne razdiobe Poissonovog procesa:]
  $$p_n(t) = \frac{(\lambda t)^n}{n!}\cdot e^{-\lambda t}, \quad n = 0,1,2,\ldots$$
  \item[Poissonov proces] $\left\{ \N_t, t \geq 0\right\}$ zadan je uvjetima:
  \begin{enumerate}
    \item $\N_0 = 0$.
    \item $\N$ ima nezavisne priraste.
    \item Slučajna varijabla $\N_t-\N_s$, $0 \leq s < t$, ima Poissonovu razdiobu s parametrom $\lambda(t-s)$, tj.:
    $$P\left\{\N_t-\N_s = k\right\} = \frac{\left[ \lambda(t-s)\right]^k}{k!}\cdot e^{-\lambda (t-s)}.$$
  \end{enumerate}

  \item[Uvjetne vjerojatnosti za Poissonov proces:] Neka je $s<t$. Za Poissonov proces vrijedi:
  $$P(\N_t = j\; \vert \; \N_s = i) = \frac{\left[\lambda(t-s)\right]^{j-i}}{(j-i)!}\cdot e^{-\lambda(t-s)}.$$

  \item[Konstrukcija Poissonovog procesa pomoću eksponencijalnih razdioba.] Neka je $(\zeta_n)$ niz nezavisnih slučajnih varijabli s eksponencijalnom razdiobom $\mathcal E(\lambda)$, koje bilježe vrijeme između uzastopnih pojavljivanja događaja $A$. Tad brojač $\N_t$ pojavljivanja događaja $A$ čini Poissonov proces s parametrom $\lambda$.

  \item[Zbroj Poissonovih procesa.] Zbroj dvaju nezavisnih Poissonovih procesa $\N_1$ i $\N_2$ s parametrima $\lambda_1$ i $\lambda_2$ je Poissonov proces s parametrom $\lambda_1 + \lambda_2$.

  \item[Poissonov proces i binomna razdioba.] Ako je $\N$ Poissonov proces te $s<t$, onda vrijedi:
  $$P(\N_s=k \vertS \N_t=n) = {n \choose k}p^k(1-p)^{n-k},\quad k=0,1,\ldots,n,$$
  pri čemu je $p=\lambda_1/\lambda_2$.

  
  \item Ako su $\N_1$ i $\N_2$ nezavisni Poissonovi procesi s parametrima $\lambda_1$ i $\lambda_2$, onda je
  $$P(\N_1(t) = k \vertS \N_1(t) + \N_2(t) = n) = {n \choose k}p^k(1-p)^{n-k},$$
  pri čemu je $p=\lambda_1/(\lambda_1+\lambda_2)$.

  \item[Poissonov proces i geometrijska razdioba.] Promatramo dva nezavisna niza događaja, $A$ i $B$ koji se pojavljuju u skladu s Poissonovim procesima s parametrima $at$, odnosno $bt$. Neka je $\N$ broj pojavljivanja događaja $A$ između dvije uzastopne realizacije događaja $B$. Onda $\N$ ima geometrijsku razdiobu.

  Vrijeme $\zeta$ između dvije uzastopne realizacije događaja $B$ ima eksponencijalnu razdiobu s gustoćom $f(x)=be^{-bx}$. Vjerojatnost da se unutar intervala $[0,t]$ događaj $A$ pojavi $k$ puta je
  $$\frac{e^{-at}(at)^k}{k!}.$$
  Zato je
  $$P(\N=k) = \frac{b}{a+b}\left(\frac{a}{a+b}\right)^k, \quad k=0,1,2,3,\ldots$$
\end{description}

\subsection{Markovljev proces}
Za slučajan proces $\X$ s vrijednostima u diskretnom skupu $S$ kažemo da je Markovljev proces, ako za sve $t_0 < t_1 < \ldots < t_n$ ima svojstvo
$$P(\X_{t_{n+1}} = x_{n+1} \vertS \X_{t_n}=x_n,\X_{t_{n-1}} = x_{n-1}, \ldots, \X_{t_0} = x_0) = P(\X_{t_{n+1}} = x_{n+1} \vertS \X_{t_n} = x_n).$$
Ako proces $\X$ ima nezavisne priraste, onda je on Markovljev proces.

\begin{description}
  \item[Vjerojatnost prijelaza:]
  $$P(\X_t=x_j \vertS \X_s = x_i)=: p(s,x_i,t,x_j) =: p_{ij}^{[s,t]}.$$
  \item[Prijelazna vjerojatnost:] $p_{ij}^{[s,t]}$ je vjerojatnost da se čestica u trenutku $t$ nađe u stanju $x_j$, ako je u trenutku $s$ bila u stanju $x_i$.
  \item[Matrica prijelaznih vjerojatnosti (matrica prijelaza):]
  $$\MP^{[s,t]} := \left(p_{ij}^{[s,t]}\right).$$
  \item[Konzervativan proces:] proces se ne može ``izgubiti,'' već se uvijek nalazi u jednom od predviđenih stanja, tj.:
  $$\sum\limits_{j=1}^\infty p_{ij}^{[s,t]} = 1, \quad \forall i.$$
\end{description}

\subsection{Chapman-Kolmogorovljeva jednadžba}
Neka je $\MP^{[s,t]}$ matrica prijelaznih vjerojatnosti Markovljevog procesa. Tada ona zadovoljava Chapman-Kolmogorovljevu jednadžbu
$$\MP^{[s,u]} = \MP^{[s,t]}\MP^{[t,u]},\quad s \leq t \leq u.$$

\noindent Poissonov proces je \emph{homogen}\footnote{Matrica prijelaza ovisi samo o razlici $t-s$.} i \emph{nestacionaran}. Formalno: $\MP^{[s,t]} =: \MP(t-s)$. Vrijedi polugrupno svojstvo:
$$P(t+s) = \MP^{[0,t+s]}=\MP^{[0,t]}\cdot \MP^{[t,t+s]}=\MP(t)\MP(t+s-t)=\MP(t)\MP(s).$$

\noindent Matrica \emph{gustoća prijelaza} $\A := (a_{ij})$, $a_{ij}:=p_{ij}'(0)$. Za \emph{prijelazne gustoće Poissonovog procesa} vrijedi:
$$p_{ij}(t)=\frac{(\lambda t)^{j-i}}{(j-i)!}\cdot e^{-\lambda t}.$$
Iz toga i $a_{ij} = p_{ij}'(0)$ dobivamo:
\begin{align}
a_{ii} &= -\lambda, \nonumber \\
a_{i,i+1} &= \lambda, \nonumber \\
a_{ij} &= 0\quad \text{za}\; j\neq i, i+1. \nonumber
\end{align}


\subsubsection{Izvod}
$$\MP(t+s)=\MP(t)\MP(s),$$
za elemente vrijedi:
$$p_{ij}(t+s) = \sum\limits_k p_{ik}(t)p_{kj}(s).$$
Označimo $a_{ij}:=p_{ij}'(0)$, te definirajmo matricu $\A := (a_{ij})$. Derivirajmo:
$$\frac{\partial}{\partial s}p_{ij}(t+s) = \sum\limits_k p_{ik}(t)\frac{\mathrm d}{\mathrm ds}p_{kj}(s).$$
Za $s=0$:
$$p_{ij}'(t)=\sum\limits_k p_{ik}(t)p_{kj}'(0)=\sum\limits_k p_{ik}(t)a_{kj}.$$
Rezultat je matrična jednadžba -- \emph{Kolmogorovljeva jednadžba unaprijed}:
$$\MP'(t)=\MP(t)\A.$$
Da smo krenuli s deriviranjem po $t$, dobili bi \emph{Kolmogorovljevu jednadžbu unazad}:
$$\MP'(t)=\A\MP(t).$$


\section{Gaussovi procesi}
\begin{description}
  \item[Normalna slučajna varijabla] $\X \sim \mathcal N(m,\sigma^2)$ zadana je funkcijom:
  $$f(x)=\frac{1}{\sigma\sqrt{2\pi}}\cdot e^{-\frac{(x-m)^2}{2\sigma^2}},$$
  pri čemu je $m$ očekivanje, a $\sigma^2$ disperzija slučajne varijable $\X$.

  \item[Koeficijent korelacije.] Neka je $(\X,\mathbf Y) \sim \mathcal N(m_1,m_2,\sigma_1^2,\sigma_2^2)$, tada je $r$ koeficijent
  korelacije varijabli $\X$ i $\mathbf Y$:
  $$r(\X,\mathbf Y) = \frac{\mathrm{cor}(\X,\mathbf Y)}{\sigma_1 \sigma_2} = \frac{\E(\X \cdot \mathbf Y) - m_1 m_2}{\sqrt{\mathbb D(\X)\mathbb D(\mathbf Y)}}.$$

  \item[Gaussov proces.] Stohastički proces $\X$ je Gaussov ako za sve $t_1 < t_2 < \ldots < t_n$ slučajni vektor
  $(\X_{t_1},\X_{t_2},\ldots,\X_{t_n})$ ima normalnu razdiobu.

  \item[Brownovo gibanje, Wienerov proces.] $\mathbf W$ je Wienerov proces ako zadovoljava uvjete:
  \begin{enumerate}
    \item $\mathbf W(0) = 0$.
    \item $\mathbf W$ ima nezavisne priraste.
    \item Prirast $\mathbf W_t - \mathbf W_s$, $s<t$ ima normalnu razdiobu $\mathcal N(\mu(t-s),\sigma^2(t-s))$.
  \end{enumerate}
\end{description}


\section{Korelacijska teorija stohastičkih procesa}
\begin{description}
  \item[Spektar od $\hat \X (u)$:]
  $$\hat \X (u) := \int\limits_{-\infty}^\infty \X(t)e^{-iut}\di t.$$
  $\hat \X$ je Fourierov transformat.
  \item[Inverz:]
  $$\X(t) = \frac{1}{2\pi}\int\limits_{-\infty}^\infty \hat\X(u)e^{iut}\di u.$$
  \item $\hat \X$ zahtjeva svojstvo \emph{apsolutne integrabilnosti}.
  \item[Odrezani proces:]
  $$\X_T(t)= \left\{
    \begin{array}{ll}
    \X(t), & -T < t < T,\\
    0, & \text{inače}.
    \end{array}
    \right.$$

  \item[Energija:]
  $$\En(T) = \int\limits_{-T}^T \X(t)^2\di t.$$
  Po Parsevalu:
  $$\En(T) = \frac{1}{2\pi}\int\limits_{-\infty}^\infty \vert \hat\X_T(u)\vert ^2\di u.$$

  \item[Snaga:]
  $$\MP(T) =  \frac{1}{2T}\int\limits_{-T}^T \X(t)^2\di t
  = \frac{1}{2\pi}\int\limits_{-\infty}^\infty \frac{\vert \hat\X_T(u)\vert ^2}{2T}\di u.$$

  \item[Prosječna snaga procesa:]
  $$\MP_{XX} = \lim\limits_{T\rightarrow \infty}\frac{1}{2T} \int\limits_{-T}^T \E\left[\X^2(t)\right]\di t = \frac{1}{2\pi}\int\limits_{-\infty}^\infty \lim\limits_{T\rightarrow \infty} \frac{\E\left[\vert\hat\X_T(u)\vert ^2\right]}{2T}\di u.$$

  \item[Vremensko usrednjenje:]
  $$\A[f(t)] = \lim\limits_{T\rightarrow \infty} \frac{1}{2T} \int\limits_{-T}^T f(t)\di t.$$
  Znači:
  $$\MP_{XX} = \A\left(\E[\X^2(t)]\right).$$

  \item[Spektralna gustoća snage procesa:]
  $$\mathbf S_{XX}(u) := \lim\limits_{T\rightarrow \infty} \frac{\E\left[\vert\hat\X_T(u)\vert^2\right]}{2T}.$$
  Znači:
  $$\MP_{XX} = \frac{1}{2\pi}\int\limits_{-\infty}^\infty \mathbf S_{XX}(u)\di u.$$
  \item[Svojstva spektralne gustoće:] \hfill
  \begin{enumerate}
    \item $\mathbf S_{XX} \geq 0$,
    \item $\mathbf S_{XX}(-u) = \mathbf S_{XX}(u)$, (ako je $\X(t)$ realan)
    \item $\mathbf S_{XX}(u)$ je realna funkcija,
    \item 
    $$\frac{1}{2\pi}\int\limits_{-\infty}^\infty \mathbf S_{XX}(u)\di u = \A\left(\E\left[\X(t)^2\right]\right),$$
    \item $\mathbf S_{XX}(u)$ i $\A[R_{XX}(t,t+\tau)]$ su par Fourierovih transformata. Vrijedi:
    $$\A[R_{XX}(t,t+\tau)] = \frac{1}{2\pi}\int\limits_{-\infty}^\infty\mathbf S_{XX}(u)e^{iu\tau}\di u,$$
    $$\mathbf S_{XX}(u) = \int\limits_{-\infty}^\infty\A[R_{XX}(t,t+\tau)] e^{-iu\tau}\di \tau.$$
  \end{enumerate}
  
  \item Ako je proces stacionaran u širem smislu, onda je $R_{XX}(t,t+\tau)=R_{XX}(\tau)$, $\forall t$, pa je i vremensko usrednjenje $\A[R_{XX}(t,t+\tau)] = \A[R_{XX}(\tau)]$. Tada vrijedi:
  $$\mathbf S_{XX}(u) = \int\limits_{-\infty}^\infty R_{XX}(\tau)e^{-iu\tau}\di \tau,$$
  $$R_{XX}(\tau) = \frac{1}{2\pi}\int\limits_{-\infty}^\infty S_{XX}(u)e^{iu\tau}\di u.$$
  $R_{XX}$ je parna funkcija, stoga je $\mathbf S_{XX}$ realna i može se zapisati u obliku \emph{kosinus transformacije}.

  \item[Kosinus transformacije:]
  \begin{align}
  \mathbf S_{XX}(u) &= \int_{-\infty}^\infty R_{XX}(\tau)e^{-iu\tau}\di \tau \nonumber\\
  &=\int_{-\infty}^\infty R_{XX}(\tau) [\cos u \tau - i \sin u\tau]\di \tau \nonumber\\
  &=\int_{-\infty}^\infty R_{XX}(\tau)\cos u \tau \di \tau -  i\int_{-\infty}^\infty R_{XX}(\tau)\sin u \tau\di \tau \nonumber\\
  &=2\int_{0}^\infty R_{XX}(\tau)\cos u \tau\di\tau \nonumber.
  \end{align}
\end{description}

%TODO
% - poisson, dovrši
% - gaussovi procesi, dovrši
% - korelacijska, dovrši
% - procjene i prognoze, sve

\end{document}
