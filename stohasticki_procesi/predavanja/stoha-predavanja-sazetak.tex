\documentclass[12pt,english]{article}
\usepackage[utf8]{inputenc}
\usepackage[croatian]{babel}
\usepackage[T1]{fontenc}
\usepackage{graphicx}
\usepackage{booktabs}
\usepackage{amsmath}
\usepackage{amssymb}
%\usepackage{times}
\usepackage{lmodern}
%\usepackage[a4paper, left=0.5cm, right=0.5cm, bottom=2cm, top=0.5cm]{geometry}
\usepackage{enumerate}

\begin{document}
\title{Stohastički procesi}
\date{}
\maketitle

\section{Funkcije izvodnice}
Funkcija izvodnica slučajne varijable $\mathbf X$:
$$\psi_\mathbf{X}(z) = \mathbb{E} (z^\mathbf X) =
\sum_{n=0}^\infty p_n z^n, \quad p_n = P(\mathbf X = n)$$

\subsection{Neke razdiobe i funkcije izvodnice}
\begin{description}
  \item[Geometrijska] (vrijeme do prve realizacije događaja)
  $$\mathbf X \sim \mathcal G(p), \quad p_k = P(\mathbf X = k) = pq^{k-1}, \quad q = 1-p.$$
  Izvodnica: $\psi(z) = \frac{p}{1-qz}$.

  \item[Binomna] (broj realizacija u $n$ pokusa)
  $$\mathbf X \sim \mathcal B(n,p), \quad p_k = P(\mathbf X = k) = {n \choose k}p^k q^{n-k}, \quad q = 1-p.$$
  Izvodnica: $\psi(z) = (pz+q)^n$.

  \item[Poissonova] (broj realizacija u nekom vremenu)
  $$\mathbf X \sim \mathcal P(\lambda), \quad p_k = P(\mathbf X = k) = \frac{\lambda^k}{k!} e^{-\lambda}.$$
  Izvodnica: $\psi(z) = e^{\lambda (z-1)}$.

\end{description}

\subsection{Očekivanje i disperzija}
Očekivanje:
$$\psi_\mathbf X'(1) = \mathbb E(\mathbf X)$$
Izvod:
$$\psi'(z) = \sum_{n=0}^\infty n z ^{n-1}p_n,$$
za $n \geq 1$ i $z = 1$ vrijedi:
$$\psi'(1) = \sum_{n=1}^\infty n p_n = \mathbb E(\mathbf X).$$
Disperzija:
$$\mathbb D(X) = \mathbb E(\mathbf X^2) - \mathbb E(\mathbf X)^2 = \psi''(1)+\psi'(1)-\psi'(1)^2.$$

\subsection{Suma slučajnih varijabli}
Ako je $\mathbf S = \mathbf X_1 + \mathbf X_2 + \ldots + \mathbf X_n$, tada vrijedi:
$$\psi_\mathbf S (z) = \psi_{\mathbf X_1}(z) \cdot \psi_{\mathbf X_2}(z) \cdots \psi_{\mathbf X_n}(z).$$
Izvod (kraći):
$$\mathbb E(z^{\mathbf X + \mathbf Y}) = \mathbb E (z^\mathbf X \cdot z^\mathbf Y) = \mathbb E(z^\mathbf X) \cdot \mathbb E(z^\mathbf Y).$$

\subsection{Slučajni zbroj slučajnih varijabli}
Ako je $\mathbf S = \mathbf X_1 + \mathbf X_2 + \cdots + \mathbf X_{\mathbf N}$, pri čemu je $\mathbf N$ slučajna varijabla, a $\mathbf X_i$ slučajne varijable jednake razdiobe, tada vrijedi:
$$\psi_{\mathbf S}(z) = \psi_{\mathbf N}(\psi_{\mathbf X}(z)).$$

\subsection{Polinomijalna razdioba}
Imamo niz od $n$ nezavisnih pokusa. U svakom pokusu se može ostvariti jedan od disjunktnih događaja $A_1, A_2, \ldots, A_r$. Vjerojatnost da se $A_1$ ostvario $n_1$ put, $A_2$ $n_2$ puta, \ldots, $A_r$ $n_r$ puta ($n_1+n_2+\cdots+n_r=n$) dana je sa:
$$p_{n_1,n_2,\ldots,n_r}=\frac{n!}{n_1!n_2!\cdots n_r!} \cdot p_1^{n_1} p_2^{n_2} \cdots p_r^{n_r}.$$
Funkcija izvodnica:
$$\psi(u_1,u_2,\ldots,u_r) = (p_1u_1+\cdots+p_ru_r)^n.$$

\subsection{Neki razvoji}


\end{document}
