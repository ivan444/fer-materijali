\documentclass{article}
\usepackage[utf8]{inputenc}
\usepackage[croatian]{babel}
\usepackage[T1]{fontenc}
\usepackage{graphicx}
\usepackage{amsmath}
\usepackage{amssymb}
\usepackage{lmodern}
\usepackage{url}
\usepackage{enumerate}
\usepackage{booktabs}
\usepackage{algorithmicx}
\usepackage[noend]{algpseudocode}
\usepackage{color}

\newcommand{\doccol}{blue}

\title{Numerička matematika}
\date{}

\begin{document}
%{\color{\doccol}
\maketitle

\section{Numerička integracija}
\begin{description}
  \item[Simpsonova formula:]
  \begin{align}
  S_f(\mathbf{x}) &= \frac{h}{3} \cdot \big [ f(\mathbf{x_0}) + f(\mathbf{x_n}) \nonumber \\
   &\quad+ 4 \cdot  ( f(\mathbf{x_1}) + f(\mathbf{x_3}) + \cdots + f(\mathbf{x_{n-1}}) ) \nonumber \\
   &\quad+ 2 \cdot ( f(\mathbf{x_2}) + f(\mathbf{x_4}) + \cdots + f(\mathbf{x_{n-2}}) ) \big ] 
  \end{align}
  Ocjena greške:
  $$E = h^4\|f^{(4)}\|_\infty \dfrac{b-a}{180}$$
  \item[Trapezna formula:]
  \begin{equation}
  T_f(\mathbf{x}) = \frac{h}{2} \cdot \left ( f(\mathbf{x_0}) + 2\sum\limits_{i=1}^{n-1} f(\mathbf{x_i}) + f(\mathbf{x_n}) \right )
  \end{equation}
  Ocjena greške:
  $$E = h^2\|f''\|_\infty \dfrac{b-a}{12}$$
  \item[Pravokutna formula:]
  \begin{equation}
  P_f(\mathbf{x}) = h \sum\limits_{i=0}^{n-1} f \left ( \frac{\mathbf{x_i} + \mathbf{x_{i+1}}}{2} \right )
  \end{equation}
  Ocjena greške:
  $$E = h^2\|f''\|_\infty \dfrac{b-a}{24}$$
\end{description}

\subsection{Analiza greške}
Osnovne mjere:
\begin{description}
  \item[apsolutna greška:] $$E_{aps} = | \tilde  x - x |,$$
  \item[relativna greška:] $$R_{rel} = \frac{E_{aps}}{|x|},$$
\end{description}
pri čemu je $\tilde x$ iznos dobiven postupkom numeričke integracije, a $x$ vrijednost dobivena analitičkim rješavanjem.

\section{Splineovi}
Uvjeti koji se moraju zadovoljiti za konstrukciju kubičnog splinea:
\begin{itemize}
  \item neprekidnost funkcije: $s(x_i-0)=s(x_i+0)$ za $i=1,2,\ldots, n-1$,
  \item interpolacija: $s(x_i)=y_i$  za $i=1,2,\ldots, n-1$,
  \item neprekidnost prve derivacije: $s'(x_i-0)=s'(x_i+0)$ za $i=1,2,\ldots, n-1$,
  \item neprekidnost druge derivacije: $s''(x_i-0)=s''(x_i+0)$ za $i=1,2,\ldots, n-1$.
\end{itemize}
Ponašanje splinea na rubovima zadanog intervala:
\begin{description}
  \item[Potpuni spline:] $s'(x_0) = f'(x_0), \quad s'(x_n) = f'(x_n)$.
  \item[Periodički spline:] $s'(x_0) = s'(x_n), \quad s''(x_0) = s''(x_n)$.
  \item[Prirodni kubični spline:] $s''(x_0) = s''(x_n) = 0$.
\end{description}

\section{Metoda najmanjih kvadrata}
Tražimo $\tilde f(x_i) \approx y_i$. Pretpostavimo oblik te u ovisnosti o koeficijentima tražimo:
$$E = \sum\limits_{i=0}^n\left (y_i-\tilde f(x_i)\right)^2 \, \rightarrow \, \text{min}.$$
Ako imamo $E(c_0, c_1, \ldots, c_n)$ tada tražimo: $$\frac{\partial E}{\partial c_i} = 0,\quad i = 0, \ldots, n.$$
Ako imamo oblik
$$\tilde f(x) = c_0 + c_1x + \ldots + c_k x^k,$$
to se može zapisati kao:
$$\begin{bmatrix}
y_0\\
y_1\\
\vdots \\
y_n
\end{bmatrix}
=
\begin{bmatrix}
1 & x_0 & x_0^2 & \cdots & x_0^k\\
1 & x_1 & x_1^2 & \cdots & x_1^k\\
\vdots & \vdots & \vdots & \ddots & \vdots\\
1 & x_n & x_n^2 & \cdots & x_n^k
\end{bmatrix}
\cdot
\begin{bmatrix}
c_0\\
c_1\\
\vdots\\
c_k
\end{bmatrix}$$

\subsection{Preodređeni sustav}
Zadan je preodređen sustav kojem je potrebno pronaći približno rješenje. Sustav možemo prikazati u obliku:
\begin{equation}
A\mathbf x = \mathbf b,
\label{eq:osnovni_sus}
\end{equation}
pri čemu je $A \in \mathbb R^{m \times n}$, $\mathbf b \in \mathbb R^m$, i $m>n$. Aproksimaciju tražimo metodom najmanjih kvadrata, odnosno tražimo $\mathbf x$ koji minimizira $\| A\mathbf x - \mathbf b\|^2$.

\subsection{Rješenje preko normalnih jednadžbi}
Normalne jednadžbe:
\begin{align}
\|A\mathbf x - b\|_2 \rightarrow \text{min} &\Leftrightarrow \left (A\mathbf x - \mathbf b, A\mathbf y \right ) = 0,\quad \forall\mathbf y \in \mathbb R^k\\
&\Leftrightarrow \left(A^\top\left(A\mathbf x - \mathbf b\right ), \mathbf y \right ) = 0,\quad \forall\mathbf y \in \mathbb R^k\\
&\Leftrightarrow A^\top \left ( A \mathbf x - \mathbf b \right ) = 0.
\label{eq:normalne}
\end{align}
Rezultat:
$$A^\top A\mathbf x = A^\top \mathbf b.$$
Time dobivamo $n$ jednadžbi sa $n$ nepoznanica, a $A^\top A$ je simetrična pozitivno definitna matrica.

\textsc{Algoritam:}
\begin{enumerate}[\indent 1)]
  \item $C = A^\top A$,
  \item primjeni Cholesky faktorizaciju na $C,\; C = LL^\top$,
  \item $\mathbf d = A^\top \mathbf b$,
  \item izračunaj $L\mathbf z = \mathbf d$ korištenjem supstitucije unaprijed,
  \item izračunaj $L^\top \mathbf x = \mathbf z$ korištenjem suspstitucije unatrag.
\end{enumerate}
Složeost postupka: $mn^2 + (1/3)n^3$ operacija.

\textsc{Cholesky faktorizacija:}
\begin{enumerate}[\indent 1)]
  \item $l_{1,1} = \sqrt{a_{1,1}}$,
  \item $l_{j,1} = a_{j,1}/l_{1,1},\quad j = 2, \ldots, n$,
  \item za $i = 2, \ldots, n-1$:
  \begin{enumerate}[\indent a)]
    \item $l_{i,i} = \left (a_{i,i} - \sum\limits_{k=1}^{i-1} l_{i,k}^2\right )^{1/2}$,
    \item $l_{j,i} = \left (a_{j,i} - \sum\limits_{k=1}^{i-1} l_{j,k}l_{i,k}\right )/l_{i,i},\quad j = i+1, \ldots, n$,
  \end{enumerate}
  \item $l_{n,n} = \left ( a_{n,n} - \sum\limits_{k=1}^{n-1} l_{n,k}^2\right ) ^{1/2}$,
\end{enumerate}
Rezultat faktorizacije je donja trokutasta matrica $L$.

\subsection{Rješenje $QR$ faktorizacijom}
Krenimo od normalnih jednadžbi (\ref{eq:normalne}) koristeći $QR$ faktorizaciju $A = QR$:
\begin{align}
A^\top A \mathbf x &= A^\top \mathbf b \nonumber \\
R^\top Q^\top Q R\mathbf x &= R^\top Q^\top\mathbf b  \nonumber \\
R^\top R\mathbf x &= R^\top Q^\top\mathbf b  \nonumber \\
 R\mathbf x &= Q^\top\mathbf b\quad (R \text{ nije singularna})\nonumber 
\end{align}
$QR$ faktorizacija matrice $A \in \mathbb R^{n \times k}$, $k<n$, je rastav oblika:
$$A = Q \begin{bmatrix}R\\0\end{bmatrix},$$
pri čemu je $Q \in \mathbb R^{n \times n}$ ortogonalna, a $R \in \mathbb R^{k\times k}$, gornjetrokutasta matrica.

\textsc{Algoritam:}
\begin{enumerate}[\indent 1)]
  \item $QR$ faktorizacija od $A$, $A = QR$,
  \item $\mathbf d = Q^\top \mathbf b$,
  \item izračunaj $R \mathbf x = \mathbf d$ korištenjem suspstitucije unatrag.
\end{enumerate}
Složenost navedenog postupka za velike $m$ i $n$ iznosi: $2mn^2$ operacija s pomičnim zarezom. 

$QR$ faktorizacijom matricu $A$ rastavljamo na umnožak $QR$ pri čemu je $Q$ ortogonalna matrica (vrijedi: $Q^\top Q = I$), a $R$ gornja trokutasta matrica. Ako matrica $A$ nije singularna, tada je $QR$ faktorizacija jedinstvena.

\textsc{$QR$ faktorizacija:}
\begin{enumerate}[\indent 1)]
  \item $\mathbf u_k = \mathbf a_k - (\mathbf a_k \cdot \mathbf e_1)\mathbf e_1 - \cdots - (\mathbf a_k \cdot \mathbf e_{k-1})\mathbf e_{k-1}, \quad k = 1, \ldots, n$,
  \item $\mathbf e_k = \frac{\mathbf u_k}{\|\mathbf u_k\|}, \quad k = 1, \ldots, n$,
  \item $Q = \begin{bmatrix} \mathbf e_1 & \mathbf e_2 & \cdots  & \mathbf e_n\end{bmatrix}$,
  \item $$R = 
  \begin{bmatrix}
  \mathbf a_1 \cdot \mathbf e_1 & \mathbf a_2 \cdot \mathbf e_1 & \cdots & \mathbf a_n \cdot \mathbf e_1 \\
  0 & \mathbf a_2 \cdot \mathbf e_2 & \cdots & \mathbf a_n \cdot \mathbf e_2 \\
  \vdots & \vdots & \ddots & \vdots \\
  0 & 0  & \cdots & \mathbf a_n \cdot \mathbf e_n
  \end{bmatrix},$$
\end{enumerate}

\section{LU dekompozicija}
\begin{align*}
A\mathbf x &= \mathbf b\\
A\mathbf x &= LU\mathbf x = \mathbf b\\
\end{align*}
Supstitucija unaprijed:
$$L\mathbf y = \mathbf b$$
$$y_i = b_i - \sum\limits_{j=1}^{i-1}l_{ij}\cdot y_j, \quad i = 1, \ldots, n$$
Supstitucija unatrag:
$$U\mathbf x = \mathbf y$$
$$x_i = \frac{1}{u_{ii}}\cdot \left ( y_i - \sum\limits_{j=i+1}^n u_{ij}\cdot x_j\right ), \quad i = n, \ldots, 1$$
LU dekompozicija:
\begin{algorithmic}
  \For{$i = 1 \textrm{ to } n-1$}
    \For{$j = i+1 \textrm{ to } n$}
        \State{$a_{ji} = a_{ji} / a_{ii}$}
        \For{$k = i+1 \textrm{ to } n$}
          \State{$a_{jk} = a_{jk}-a_{ji}\cdot a_{ik}$}
        \EndFor
    \EndFor
  \EndFor
\end{algorithmic}
Matrica ima jedinstveno LU dekompoziciju ako su sve njene glavne podmatrice regularne.

\section{Nelinearne jednadžbe}
\begin{description}
  \item[Broj točnih znamenki:] $$\left\lfloor -\log_{10} \left ( \frac{|x^{(k)} - x^\ast |}{|x^\ast|}\right ) \right \rfloor$$
\end{description}

\subsection{Metoda jednostavnih iteracija}
Nelinearni jednadžbu $f(x) = 0$ zapišemo u obliku $x = \varphi (x)$.
\begin{algorithmic}
\Require{$x_0$ i $\varepsilon$}
\While{$|x_{n+1} - x_n| < \varepsilon$}
  \State{$x_{n+1} = \varphi(x_n)$}
  \State{n = n+1}
\EndWhile
\end{algorithmic}


\subsection{Newton}
\begin{algorithmic}
\Require{$x_0$ i $\varepsilon$} \Comment{$x_0$ je blizu rješenja!}
\While{$|f(x)| > \varepsilon$}
  \State{$x = x- f(x)/f'(x)$}
\EndWhile
\end{algorithmic}
\textsc{Kriteriji zaustavljanja:}
\begin{description}
  \item[Ocjena reziduala:] $|f(x_n)| \leq \varepsilon |f(x_0)|,$
  \item[Ocjena duljine koraka:] $|x_{n+1}-x_n| \leq \varepsilon |x_0|.$
\end{description}

\subsection{Metoda sekante}
\begin{algorithmic}
\Require{$x_0$, $x_p \neq x_0$ i $\varepsilon$}
\While{$|f(x)| > \varepsilon$}
  \State{$g = \frac{f(x)-f(x_p)}{x-x_p}$}
  \State{$x_p = x$}
  \State{$x = x- f(x)/g$}
\EndWhile
\end{algorithmic}

\section{Newton za sustave nelinearnih jednadžbi}
\begin{align*}
f_1(x_1, \ldots&, x_n) = 0 \\
f_2(x_1, \ldots&, x_n) = 0 \\
\vdots&\\
f_n(x_1, \ldots&, x_n) = 0
\end{align*}
Imamo $n$ nelinearnih jednadžbi sa $n$ nepoznanica.

Jakobijan:
$$D\mathbf f(\mathbf x) =
\begin{bmatrix}
\dfrac{\partial f_1(\mathbf x)}{\partial x_1} & \dfrac{\partial f_1(\mathbf x)}{\partial x_2} & \cdots & \dfrac{\partial f_1(\mathbf x)}{\partial x_n} \\ 
\dfrac{\partial f_2(\mathbf x)}{\partial x_1} & \dfrac{\partial f_2(\mathbf x)}{\partial x_2} & \cdots & \dfrac{\partial f_2(\mathbf x)}{\partial x_n} \\ 
\vdots &  \vdots & \ddots & \vdots \\
\dfrac{\partial f_n(\mathbf x)}{\partial x_1} & \dfrac{\partial f_n(\mathbf x)}{\partial x_2} & \cdots & \dfrac{\partial f_n(\mathbf x)}{\partial x_n}  
\end{bmatrix}$$
\textsc{Algoritam:}
\begin{algorithmic}
\Require{$x$ i $\varepsilon$}
\While{$\|\mathbf f(\mathbf x)\| > \varepsilon$}
  \State{$x = x- D\mathbf f(\mathbf x)^{-1}\mathbf f(\mathbf x) $}
\EndWhile
\end{algorithmic}
Pretpostavimo da je $D\mathbf f(\mathbf x)$ nesingularna. Računaj:
\begin{align*}
H = D\mathbf f(\mathbf x) \\
g = \mathbf f(\mathbf x)
\end{align*}
riješi:
\begin{align*}
H\Delta x = -g \\
x = x+\Delta x
\end{align*}

\subsection{Globalno konvergentna Newtonova metoda}
\begin{algorithmic}
\Require{$x_0$}
\For{$n = 0, 1, \ldots$}
  \State{$x_{n+1} = x_n - f(x_n)/f'(x_n)$}
  \While{$|f(x_{n+1})| \geq |f(x_n)|$}
    \State{$x_{n+1} = (x_{n+1} + x_n)/2$}
  \EndWhile
\EndFor
\end{algorithmic}

\section{IEEE fp format}
Predznak $z$ (1 bit), eksponent $e$ ($k$ bitova), frakcija $f$ ($p$ bitova); $a = 2^{k-1}-1$.
\begin{table}[!h]
\centering
%{\color{\doccol}
\caption{Izračun fp brojeva}
\begin{tabular}{ccc}
\toprule
$e$ & $f$ & Vrijednost \\
\midrule
0 & 0 & $(-1)^z \cdot 0 = \pm 0$\\
0 & $1\le f \le 2^p-1$ & $(-1)^z \cdot 2^{-a+1}\cdot(0.f)$\\
$1\le e \le 2^k-2$ & $0\le f \le 2^p-1$ & $(-1)^z \cdot 2^{e-a}\cdot(1.f)$\\
$2^{k}-1$ & 0 & $(-1)^z \cdot \infty$\\
$2^{k}-1$ & $\neq 0$ & NaN\\
\bottomrule
\end{tabular}
%}
\end{table}

\begin{description}
  \item[Strojni epsilon:] udaljenost između 1 i prvog većeg fp-broja. Određen je sa
  $$\varepsilon = 2^{-p},$$
  pri čemu je $p$ broj bitova mantise.
  \item[Maks.\ pogreška u bin.\ bazi:] $2^{k-p-1}$
  \item[Jednostruka preciznost:] 1 bit za predznak, 8 za eksponent i 23 za mantisu.
  \item[Dvostruka preciznost:] 1 bit za predznak, 11 za eksponent i 52 za mantisu.
\end{description}

\section{Asimptotsko ponašanje funkcija}
$$f = O(g) (x \rightarrow x_0) \quad \Leftrightarrow \quad \lim\limits_{x \rightarrow x_0} \sup \left | \frac{f(x)}{g(x)}\right| < \infty$$
$$f = o(g) (x \rightarrow x_0) \quad \Leftrightarrow \quad \lim\limits_{x \rightarrow x_0} \left | \frac{f(x)}{g(x)}\right| = 0$$

\section{Linearna algebra}
\begin{description}
  \item[Singularnost:] Matrica $A \in \mathbb R^{n \times n}$ je singularna te vrijedi:
  \begin{itemize}
    \item $A$ nema inverz,
    \item $\textrm{det}(A) = 0$,
    \item $\textrm{rang}(A) < n$,
    \item $A \mathbf x = 0$ ima netrivijalno rješenje.
  \end{itemize}
  \item[Regularnost:] Matrica $A \in \mathbb R^{n \times n}$ je regularna te vrijedi:
  \begin{itemize}
    \item $A$ nema inverz,
    \item $\textrm{rang}(A) = n$,
    \item $A \mathbf x = 0 \Leftrightarrow \mathbf x = 0$, 
    \item $A \mathbf x = \mathbf b$ ima jedinstveno rješenje.
  \end{itemize}
  \item[Svojstvene vrijednosti:]
  $$\textrm{det}(A-\lambda I) = 0$$
  \item[Pozitivno definitna matrica:]
  $$\mathbf x^\top A \mathbf x > 0, \quad \forall \mathbf x \neq 0.$$
  \item[Determinanta:]
  $$\textrm{det}(A) = \left\{
    \begin{array}{@{\>}l@{\quad}l}
    a_{11}, & n = 1,\\
    \sum\limits_{j=1}^n \Delta_{ij}\cdot a_{ij}, & n > 1, \forall i = 1,\ldots ,n
    \end{array}
    \right.$$
  pri čemu je
  $$\Delta_{ij} = (-1)^{i+j} \textrm{det}(A_{ij}),$$
  dok je $A_{ij}$ matrica $A$ bez $i$-tog retka i $j$-tog stupca.
  \item[Uvjetovanost simetrične pozitivno definitne matrice:] $\kappa(A) = \|A\|\|A^{-1}\|$, matrica je loše uvjetovana kad vrijedi $\kappa(A)\gg 1$.
  \item[Ortogonalna transformacija vektora:] vektor $Qx$ pri čemu je $Q$ ortogonalna matrica ($Q^\top Q = I$). Ortogonalne transformacije čuvaju euklidsku normu: $\|Qx\|^2_2 = (Qx)^\top Qx = x^\top Q^\top Qx = x^\top x = \|x\|^2_2$.

\end{description}

%}
\end{document}

